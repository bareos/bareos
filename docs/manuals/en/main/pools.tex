
\chapter{Automated Disk Backup}
\label{PoolsChapter}
\index[general]{Volumes!Using Pools to Manage}
\index[general]{Disk!Automated Backup}
\index[general]{Automated Disk Backup}
\index[general]{Pool}

If you manage five or ten machines and have a nice tape backup, you don't need
Pools, and you may wonder what they are good for. In this chapter, you will
see that Pools can help you optimize disk storage space. The same techniques
can be applied to a shop that has multiple tape drives, or that wants to mount
various different Volumes to meet their needs.

The rest of this chapter will give an example involving backup to disk
Volumes, but most of the information applies equally well to tape Volumes.

Given is a scenario, where the size of a full backup is about 15GB.

It is required, that backup data is available for six months.
Old files should be available on a daily basis for a week, a weekly basis for a month, then monthly
for six months. In addition, offsite capability is not needed.
The daily changes amount to about
300MB on the average, or about 2GB per week.

As a consequence, the total volume of data they need to keep to meet their
needs is about 100GB (15GB x 6 + 2GB x 5 + 0.3 x 7) = 102.1GB.

The chosen solution was to use a 120GB hard disk -- far
less than 1/10th the price of a tape drive and the cassettes to handle the
same amount of data, and to have the backup software write to disk files.

The rest of this chapter will explain how to setup Bareos so that it would
automatically manage a set of disk files with the minimum sysadmin
intervention.

\section{Overall Design}
\label{OverallDesign}

Getting Bareos to write to disk rather than tape in the simplest case is
rather easy.

One needs to consider about what happens if we have only a single large Bareos
Volume defined on our hard disk. Everything works fine until the Volume fills,
then Bareos will ask you to mount a new Volume. This same problem applies to
the use of tape Volumes if your tape fills. Being a hard disk and the only one
you have, this will be a bit of a problem. It should be obvious that it is
better to use a number of smaller Volumes and arrange for Bareos to
automatically recycle them so that the disk storage space can be reused.

As mentioned, the solution is to have multiple Volumes, or files on the disk.
To do so, we need to limit the use and thus the size of a single Volume, by
time, by number of jobs, or by size. Any of these would work, but we chose to
limit the use of a single Volume by putting a single job in each Volume with
the exception of Volumes containing Incremental backup where there will be 6
jobs (a week's worth of data) per volume. The details of this will be
discussed shortly.  This is a single client backup, so if you have multiple
clients you will need to multiply those numbers by the number of clients,
or use a different system for switching volumes, such as limiting the
volume size.

\TODO{This chapter will get rewritten. Instead of limiting a Volume to one job, we will utilize \variable{Max Use Duration = 24 hours}. This prevents problems when adding more clients, because otherwise each job has to run seperat.}

The next problem to resolve is recycling of Volumes. As you noted from above,
the requirements are to be able to restore monthly for 6 months, weekly for a
month, and daily for a week. So to simplify things, why not do a Full save
once a month, a Differential save once a week, and Incremental saves daily.
Now since each of these different kinds of saves needs to remain valid for
differing periods, the simplest way to do this (and possibly the only) is to
have a separate Pool for each backup type.

The decision was to use three Pools: one for Full saves, one for Differential
saves, and one for Incremental saves, and each would have a different number
of volumes and a different Retention period to accomplish the requirements.

\label{FullPool}
\subsection{Full Pool}
\index[general]{Pool!Full}
\index[general]{Full Pool}

Putting a single Full backup on each Volume, will require six Full save
Volumes, and a retention period of six months. The Pool needed to do that is:

\begin{bconfig}{Full-Pool}
Pool {
  Name = Full-Pool
  Pool Type = Backup
  Recycle = yes
  AutoPrune = yes
  Volume Retention = 6 months
  Maximum Volume Jobs = 1
  Label Format = Full-
  Maximum Volumes = 9
}
\end{bconfig}

Since these are disk Volumes, no space is lost by having separate Volumes for
each backup (done once a month in this case). The items to note are the
retention period of six months (i.e. they are recycled after six months), that
there is one job per volume (Maximum Volume Jobs = 1), the volumes will be
labeled Full-0001, ... Full-0006 automatically. One could have labeled these
manually from the start, but why not use the features of Bareos.

Six months after the first volume is used, it will be subject to pruning
and thus recycling, so with a maximum of 9 volumes, there should always be
3 volumes available (note, they may all be marked used, but they will be
marked purged and recycled as needed).

If you have two clients, you would want to set {\bf Maximum Volume Jobs} to
2 instead of one, or set a limit on the size of the Volumes, and possibly
increase the maximum number of Volumes.


\label{DiffPool}
\subsection{Differential Pool}
\index[general]{Pool!Differential}
\index[general]{Differential Pool}

For the Differential backup Pool, we choose a retention period of a bit longer
than a month and ensure that there is at least one Volume for each of the
maximum of five weeks in a month. So the following works:

\begin{bconfig}{Differential Pool}
Pool {
  Name = Diff-Pool
  Pool Type = Backup
  Recycle = yes
  AutoPrune = yes
  Volume Retention = 40 days
  Maximum Volume Jobs = 1
  Label Format = Diff-
  Maximum Volumes = 10
}
\end{bconfig}

As you can see, the Differential Pool can grow to a maximum of 9 volumes,
and the Volumes are retained 40 days and thereafter they can be recycled. Finally
there is one job per volume. This, of course, could be tightened up a lot, but
the expense here is a few GB which is not too serious.

If a new volume is used every week, after 40 days, one will have used 7
volumes, and there should then always be 3 volumes that can be purged and
recycled.

See the discussion above concering the Full pool for how to handle multiple
clients.

\label{IncPool}
\subsection{Incremental Pool}
\index[general]{Incremental Pool}
\index[general]{Pool!Incremental}

Finally, here is the resource for the Incremental Pool:

\begin{bconfig}{Incremental Pool}
Pool {
  Name = Inc-Pool
  Pool Type = Backup
  Recycle = yes
  AutoPrune = yes
  Volume Retention = 20 days
  Maximum Volume Jobs = 6
  Label Format = Inc-
  Maximum Volumes = 7
}
\end{bconfig}

We keep the data for 20 days rather than just a week as the needs require. To
reduce the proliferation of volume names, we keep a week's worth of data (6
incremental backups) in each Volume. In practice, the retention period should
be set to just a bit more than a week and keep only two or three volumes
instead of five. Again, the lost is very little and as the system reaches the
full steady state, we can adjust these values so that the total disk usage
doesn't exceed the disk capacity.

If you have two clients, the simplest thing to do is to increase the
maximum volume jobs from 6 to 12. As mentioned above, it is also possible
limit the size of the volumes.  However, in that case, you will need to
have a better idea of the volume or add sufficient volumes to the pool so
that you will be assured that in the next cycle (after 20 days) there is
at least one volume that is pruned and can be recycled.


\section{Configuration Files}

The following example shows you the actual files used, with only a few minor
modifications to simplify things.

The Director's configuration file is as follows:

\begin{bconfig}{bareos-dir.conf}
Director {          # define myself
  Name = bareos-dir
  QueryFile = "/usr/lib/bareos/scripts/query.sql"
  Maximum Concurrent Jobs = 1
  Password = "*** CHANGE ME ***"
  Messages = Standard
}

JobDefs {
  Name = "DefaultJob"
  Type = Backup
  Level = Incremental
  Client = bareos-fd
  FileSet = "Full Set"
  Schedule = "WeeklyCycle"
  Storage = File
  Messages = Standard
  Pool = Inc-Pool
  Full Backup Pool = Full-Pool
  Incremental Backup Pool = Inc-Pool
  Differential Backup Pool = Diff-Pool
  Priority = 10
  Write Bootstrap = "/var/lib/bareos/%c.bsr"
}

Job {
  Name = client
  Client = client-fd
  JobDefs = "DefaultJob"
  FileSet = "Full Set"
}

# Backup the catalog database (after the nightly save)
Job {
  Name = "BackupCatalog"
  Client = client-fd
  JobDefs = "DefaultJob"
  Level = Full
  FileSet="Catalog"
  Schedule = "WeeklyCycleAfterBackup"
  # This creates an ASCII copy of the catalog
  # Arguments to make_catalog_backup.pl are:
  #  make_catalog_backup.pl <catalog-name>
  RunBeforeJob = "/usr/lib/bareos/scripts/make_catalog_backup.pl MyCatalog"
  # This deletes the copy of the catalog
  RunAfterJob  = "/usr/lib/bareos/scripts/delete_catalog_backup"
  # This sends the bootstrap via mail for disaster recovery.
  # Should be sent to another system, please change recipient accordingly
  Write Bootstrap = "|/usr/sbin/bsmtp -h localhost -f \"\(Bareos\) \" -s \"Bootstrap for Job %j\" root@localhost"
  Priority = 11                   # run after main backup
}

# Standard Restore template, to be changed by Console program
Job {
  Name = "RestoreFiles"
  Type = Restore
  Client = client-fd
  FileSet="Full Set"
  Storage = File
  Messages = Standard
  Pool = Default
  Where = /tmp/bareos-restores
}

# List of files to be backed up
FileSet {
  Name = "Full Set"
  Include = {
    Options {
      signature=SHA1;
      compression=GZIP9
    }
    File = /
    File = /usr
    File = /home
    File = /boot
    File = /var
    File = /opt
  }
  Exclude = {
    File = /proc
    File = /tmp
    File = /.journal
    File = /.fsck
    ...
  }
}

Schedule {
  Name = "WeeklyCycle"
  Run = Level=Full 1st sun at 2:05
  Run = Level=Differential 2nd-5th sun at 2:05
  Run = Level=Incremental mon-sat at 2:05
}

# This schedule does the catalog. It starts after the WeeklyCycle
Schedule {
  Name = "WeeklyCycleAfterBackup"
  Run = Level=Full sun-sat at 2:10
}

# This is the backup of the catalog
FileSet {
  Name = "Catalog"
  Include {
    Options {
      signature = MD5
    }
    File = "/var/lib/bareos/bareos.sql" # database dump
    File = "/etc/bareos"                # configuration
  }
}

Client {
  Name = client-fd
  Address = client
  FDPort = 9102
  Password = " *** CHANGE ME ***"
  AutoPrune = yes      # Prune expired Jobs/Files
  Job Retention = 6 months
  File Retention = 60 days
}

Storage {
  Name = File
  Address = localhost
  Password = " *** CHANGE ME ***"
  Device = FileStorage
  Media Type = File
}

Catalog {
  Name = MyCatalog
  dbname = bareos; user = bareos; password = ""
}

Pool {
  Name = Full-Pool
  Pool Type = Backup
  Recycle = yes           # automatically recycle Volumes
  AutoPrune = yes         # Prune expired volumes
  Volume Retention = 6 months
  Maximum Volume Jobs = 1
  Label Format = Full-
  Maximum Volumes = 9
}

Pool {
  Name = Inc-Pool
  Pool Type = Backup
  Recycle = yes           # automatically recycle Volumes
  AutoPrune = yes         # Prune expired volumes
  Volume Retention = 20 days
  Maximum Volume Jobs = 6
  Label Format = Inc-
  Maximum Volumes = 7
}

Pool {
  Name = Diff-Pool
  Pool Type = Backup
  Recycle = yes
  AutoPrune = yes
  Volume Retention = 40 days
  Maximum Volume Jobs = 1
  Label Format = Diff-
  Maximum Volumes = 10
}

Messages {
  Name = Standard
  mailcommand = "bsmtp -h mail.domain.com -f \"\(Bareos\) %r\"
      -s \"Bareos: %t %e of %c %l\" %r"
  operatorcommand = "bsmtp -h mail.domain.com -f \"\(Bareos\) %r\"
      -s \"Bareos: Intervention needed for %j\" %r"
  mail = root@domain.com = all, !skipped
  operator = root@domain.com = mount
  console = all, !skipped, !saved
  append = "/home/bareos/bin/log" = all, !skipped
}
\end{bconfig}

and the Storage daemon's configuration file is:

\begin{bconfig}{bareos-sd.conf}
Storage {               # definition of myself
  Name = bareos-sd
}

Director {
  Name = bareos-dir
  Password = " *** CHANGE ME ***"
}

Device {
  Name = FileStorage
  Media Type = File
  Archive Device = /var/lib/bareos/storage
  LabelMedia = yes;    # lets Bareos label unlabeled media
  Random Access = yes;
  AutomaticMount = yes;   # when device opened, read it
  RemovableMedia = no;
  AlwaysOpen = no;
}

Messages {
  Name = Standard
  director = bareos-dir = all
}
\end{bconfig}
