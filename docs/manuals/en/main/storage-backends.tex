\chapter{Storage Backends}

A Bareos Storage Daemon can use various storage backends:

\begin{description}
\item [\sdBackend{Tape}{}] is used to access tape device and thus has sequential access.
\item [\sdBackend{File}{}]
  tells Bareos that the device is a file. It may either be a
  file defined on fixed medium or a removable filesystem such as
  USB.  All files must be random access devices.
\item [\sdBackend{Fifo}{}] is a first-in-first-out sequential access read-only
  or write-only device.
\item [\sdBackend{Droplet}{}] is used to access an object store supported by \package{libdroplet}, most notably S3.
  For details, refer to \nameref{SdBackendDroplet}.
\item [\sdBackend{GFAPI}{GlusterFS}] is used to access a GlusterFS storage.
\item [\sdBackend{Rados}{Ceph Object Store}] is used to access a Ceph object store.
\end{description}


\input{storage-backend-droplet}


\section{GFAPI Storage Backend}
\label{SdBackendGfapi}

\sdBackend{GFAPI}{GlusterFS}

A GlusterFS Storage can be used as Storage backend of Bareos.
Prerequistes are a working GlusterFS storage system and the package \package{bareos-storage-glusterfs}.
See \url{http://www.gluster.org/} for more information regarding GlusterFS installation and configuration
and specifically \url{https://gluster.readthedocs.org/en/latest/Administrator Guide/Bareos/}
for Bareos integration.
You can use following snippet to configure it as storage device:
\bconfigInput{config/SdDeviceDeviceOptionsGfapi1.conf}
Adapt server and volume name to your environment.

\sinceVersion{sd}{GlusterFS Storage}{15.2.0}



\section{Rados Storage Backend}
\label{SdBackendRados}

\sdBackend{Rados}{Ceph Object Store}

Here you configure the Ceph object store, which is accessed by the SD using the Rados library.
Prerequistes are a
working Ceph object store and the package \package{bareos-storage-ceph}.
See \url{http://ceph.com} for more information regarding Ceph installation and configuration.
Assuming that you have an object store with name \file{poolname}
and your Ceph access is configured in \file{/etc/ceph/ceph.conf},
you can use following snippet to configure it as storage device:
\bconfigInput{config/SdDeviceDeviceOptionsRados1.conf}

\sinceVersion{sd}{Ceph Storage}{15.2.0}
