%%
%%

\chapter{Supported Tape Drives}
\label{SupportedDrives}
\index[general]{Drives!Supported Tape}
\index[general]{Supported Tape Drives}

Bareos uses standard operating system calls (read, write, ioctl) to
interface to tape drives.  As a consequence, it relies on having a
correctly written OS tape driver. Bareos is known to work perfectly well
with SCSI tape drivers on FreeBSD, Linux, Solaris, and Windows machines,
and it may work on other *nix machines.

Recently there are many new drives that use IDE, ATAPI, or
SATA interfaces rather than SCSI. On Linux the OnStream drive, which uses
the OSST driver is one such
example, and it is known to work with Bareos. In addition a number of such
tape drives (i.e. OS drivers) seem to work on Windows systems.  However,
non-SCSI tape drives (other than the OnStream) that use ide-scis, ide-tape,
or other non-scsi drivers do not function correctly with Bareos (or any
other demanding tape application) as of today (April 2007).  If you
have purchased a non-SCSI tape drive for use with Bareos on Linux, there
is a good chance that it will not work. We are working with the kernel
developers to rectify this situation, but it will not be resolved in the
near future.

Generally any modern tape drive (i.e. after 2010) will work out
of the box with Bareos using the standard Bareos Device specification
in the bareos-sd.conf file.

Even if your drive is on the list below, please check the
\ilink{Tape Testing Chapter}{btape1} of this manual for
procedures that you can use to verify if your tape drive will work with
Bareos. If your drive is in fixed block mode, it may appear to work with
Bareos until you attempt to do a restore and Bareos wants to position the
tape. You can be sure only by following the procedures suggested above and
testing.

It is very difficult to supply a list of supported tape drives, or drives that
are known to work with Bareos because of limited feedback (so if you use
Bareos on a different drive, please let us know). Based on user feedback, the
following drives are known to work with Bareos. A dash in a column means
unknown:

\addcontentsline{lot}{table}{Supported Tape Drives}
\begin{longtable}{|p{2.0in}|l|l|p{2.5in}|l|}
 \hline
\multicolumn{1}{|c| }{\bf OS } & \multicolumn{1}{c| }{\bf Man. } &
\multicolumn{1}{c| }{\bf Media } & \multicolumn{1}{c| }{\bf Model } &
\multicolumn{1}{c| }{\bf Capacity  } \\
 \hline {- } & {ADIC } & {DLT } & {Adic Scalar 100 DLT } & {100GB  } \\
 \hline {- } & {ADIC } & {DLT } & {Adic Fastor 22 DLT } & {-  } \\
 \hline {FreeBSD 5.4-RELEASE-p1 amd64 } & {Certance} & {LTO } & {AdicCertance CL400 LTO Ultrium 2 } & {200GB  } \\
 \hline {- } & {- } & {DDS } & {Compaq DDS 2,3,4 } & {-  } \\
 \hline {SuSE 8.1 Pro} & {Compaq} & {AIT } & {Compaq AIT 35 LVD } & {35/70GB } \\
 \hline {- } & {HP } & {Travan 4 } & {Colorado T4000S } & {-  } \\
 \hline {- } & {HP } & {DLT } & {HP DLT drives } & {-  } \\
 \hline {- } & {HP } & {LTO } & {HP LTO Ultrium drives } & {-  } \\
 \hline {- } & {IBM} & {??} & {3480, 3480XL, 3490, 3490E, 3580 and 3590 drives} & {-  } \\
 \hline {FreeBSD 4.10 RELEASE } & {HP } & {DAT } & {HP StorageWorks DAT72i } & {-  } \\
 \hline {- } & {Overland } & {LTO } & {LoaderXpress LTO } & {-  } \\
 \hline {- } & {Overland } & {- } & {Neo2000 } & {-  } \\
 \hline {- } & {OnStream } & {- } & {OnStream drives (see below) } & {-  } \\
 \hline {FreeBSD 4.11-Release} & {Quantum } & {SDLT } & {SDLT320 } & {160/320GB  } \\
 \hline {- } & {Quantum } & {DLT } & {DLT-8000 } & {40/80GB  } \\
 \hline {Linux } & {Seagate } & {DDS-4 } & {Scorpio 40 } & {20/40GB  } \\
 \hline {FreeBSD 4.9 STABLE } & {Seagate } & {DDS-4 } & {STA2401LW } & {20/40GB  } \\
 \hline {FreeBSD 5.2.1 pthreads patched RELEASE } & {Seagate } & {AIT-1 } & {STA1701W} & {35/70GB  } \\
 \hline {Linux } & {Sony } & {DDS-2,3,4 } & {- } & {4-40GB  } \\
 \hline {Linux } & {Tandberg } & {- } & {Tandbert MLR3 } & {-  } \\
 \hline {FreeBSD } & {Tandberg } & {- } & {Tandberg SLR6 } & {-  } \\
 \hline {Solaris } & {Tandberg } & {- } & {Tandberg SLR75 } & {- } \\
 \hline

\end{longtable}

There is a list of \ilink{supported autochangers}{Models} in the Supported
Autochangers chapter of this document, where you will find other tape drives
that work with Bareos.

\section{Unsupported Tape Drives}
\label{UnSupportedDrives}
\index[general]{Unsupported Tape Drives}
\index[general]{Drives!Unsupported Tape}

Previously OnStream IDE-SCSI tape drives did not work with Bareos. As of
Bareos version 1.33 and the osst kernel driver version 0.9.14 or later, they
now work. Please see the testing chapter as you must set a fixed block size.

QIC tapes are known to have a number of particularities (fixed block size, and
one EOF rather than two to terminate the tape). As a consequence, you will
need to take a lot of care in configuring them to make them work correctly
with Bareos.

\section{FreeBSD Users Be Aware!!!}
\index[general]{FreeBSD Users Be Aware}
\index[general]{Aware!FreeBSD Users Be}

Unless you have patched the pthreads library on FreeBSD 4.11 systems, you will
lose data when Bareos spans tapes. This is because the unpatched pthreads
library fails to return a warning status to Bareos that the end of the tape is
near. This problem is fixed in FreeBSD systems released after 4.11. Please see the
\ilink{Tape Testing Chapter}{FreeBSDTapes} of this manual for
{\bf important} information on how to configure your tape drive for
compatibility with Bareos.

\section{Supported Autochangers}
\index[general]{Autochangers!Supported}
\index[general]{Supported Autochangers}

For information on supported autochangers, please see the
\ilink{Autochangers Known to Work with Bareos}{Models}
section of the Supported Autochangers chapter of this manual.

\section{Tape Specifications}
\index[general]{Specifications!Tape}
\index[general]{Tape Specifications}
If you want to know what tape drive to buy that will work with Bareos,
we really cannot tell you.  However, we can say that if you are going
to buy a drive, you should try to avoid DDS drives. The technology is
rather old and DDS tape drives need frequent cleaning.  DLT drives are
generally much better (newer technology) and do not need frequent
cleaning.

Below, you will find a table of DLT and LTO tape specifications that will
give you some idea of the capacity and speed of modern tapes. The
capacities that are listed are the native tape capacity without compression.
All modern drives have hardware compression, and manufacturers often list
compressed capacity using a compression ration of 2:1. The actual compression
ratio will depend mostly on the data you have to backup, but I find that
1.5:1 is a much more reasonable number (i.e. multiply the value shown in
the table by 1.5 to get a rough average of what you will probably see).
The transfer rates are rounded to the nearest GB/hr.  All values are provided
by various manufacturers.

The Media Type is what is designated by the manufacturers and you are not
required to use (but you may) the same name in your Bareos conf resources.


\begin{tabular}{|c|c|c|c}
Media Type      & Drive Type & Media Capacity & Transfer Rate \\ \hline
DDS-1              & DAT        & 2 GB &        ?? GB/hr   \\ \hline
DDS-2              & DAT        & 4 GB &        ?? GB/hr   \\ \hline
DDS-3              & DAT        & 12 GB &       5.4 GB/hr   \\ \hline
Travan 40          & Travan     & 20 GB &       ?? GB/hr    \\ \hline
DDS-4              & DAT        & 20 GB &       11 GB/hr    \\ \hline
VXA-1              & Exabyte    & 33 GB &       11 GB/hr    \\ \hline
DAT-72             & DAT        & 36 GB &       13 GB/hr    \\ \hline
DLT IV             & DLT8000    & 40 GB  &      22 GB/hr    \\ \hline
VXA-2              & Exabyte    & 80 GB &       22 GB/hr    \\ \hline
Half-high Ultrium 1 & LTO 1      & 100 GB &      27 GB/hr    \\ \hline
Ultrium 1          & LTO 1      & 100 GB &      54 GB/hr    \\ \hline
Super DLT 1        & SDLT 220   & 110 GB &      40 GB/hr    \\ \hline
VXA-3              & Exabyte    & 160 GB &      43 GB/hr    \\ \hline
Super DLT I        & SDLT 320   & 160 GB &      58 GB/hr    \\ \hline
Ultrium 2          & LTO 2      & 200 GB &      108 GB/hr   \\ \hline
Super DLT II       & SDLT 600   & 300 GB &      127 GB/hr   \\ \hline
VXA-4              & Exabyte    & 320 GB &      86 GB/hr    \\ \hline
Ultrium 3          & LTO 3      & 400 GB &      216 GB/hr   \\ \hline
\end{tabular}
