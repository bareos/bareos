%%
%%

\chapter{La commande restore de la console Bacula}
\label{_ChapterStart13}
\index[general]{Commande!restore de la console Bacula}
\index[general]{La commande restore de la console Bacula}
\addcontentsline{toc}{section}{La commande restore de la console Bacula}

\section{G\'en\'eralit\'es}
\index[general]{G\'en\'eralit\'es}
\addcontentsline{toc}{section}{G\'en\'eralit\'es}

Nous allons maintenant d\'ecrire la restauration de fichiers avec la commande
{\bf restore} de la Console, qui est le mode de restauration recommand\'e.
Il existe cependant un programme ind\'ependant nomm\'e {\bf bextract}, qui permet
lui aussi de restaurer des fichiers. Pour plus d'informations sur ce
programme, consultez le chapitre \ilink{Programmes utilitaires Bacula}{bextract}
de ce manuel. Vous y trouverez aussi des informations sur le programme {\bf bls}
qui sert \`a produire une liste du contenu de vos volumes, et sur le programme
{\bf bscan} qui vous sera utilie si vous voulez restaurer les enregistrements
du catalogue relatifs \`a un ancien volume qui n'y figure plus.

En g\'en\'eral, pour restaurer un fichier ou un ensemble de fichiers, vous devez
ex\'ecuter un job de type {\bf restore}, par cons\'equent, vous devez pr\'ed\'efinir
un tel job dans le fichier de configuration de votre Director. Les param\`etres
(Client, FileSet,...) que vous d\'efinissez ici ne sont pas importants,
Bacula les ajustera automatiquement lors de l'utilisation de {\bf restore}.

Bacula \'etant un programme r\'eseau, il vous appartient de vous assurer que
vous avez s\'electionn\'e le bon client et le bon disque dur pour recevoir la
restauration. Bacula peut sauvegarder le client A et restaurer ses fichiers 
sur le client B, pourvu que leurs syst\`emes ne soient pas trop diff\'erents 
au niveau de leurs structures de fichiers. Par d\'efaut, Bacula restaure les 
donn\'ees sur leur client d'origine, mais pas \`a leur emplacement d'origine : 
dans le r\'epertoire {\bf /tmp/bacula-restores}. Vous pouvez modifier ces 
valeurs par d\'efaut lorsque la commande {\bf restore} vous demande confirmation 
d'ex\'ecution du job en choisissant l'option {\bf mod}.

\label{Example1}
\section{La commande Restore}
\index[general]{Commande!Restore }
\index[general]{La commande Restore}
\addcontentsline{toc}{section}{La commande Restore}
Puisque Bacula maintient un catalogue des fichiers sauvegard\'es, et des volumes 
o\`u ils sont stock\'es, il peut se charger de la majeure partie du travail 
d'intendance. Ainsi, il vous suffit de sp\'ecifier le type de restauration que 
vous souhaitez (d'apr\`es la derni\`ere sauvegarde, d'apr\`es la derni\`ere sauvegarde 
ant\'erieure \`a une date sp\'ecifi\'ee...), et quels fichiers vous voulez restaurer. 

Ceci est r\'ealis\'e par la commande {\bf restore} de la Console. Vous s\'electionnez 
d'abord le type de restauration souhait\'ee ce qui entra\^ine la s\'election des 
JobIds requis et la construction d'une arborescence interne \`a Bacula contenant 
les enregistrements de fichiers des JobIds s\'electionn\'es. A ce stade, le 
processus de restauration entre dans un mode o\`u vous pouvez naviguer 
interactivement dans l'arborescence des fichiers disponibles pour restauration 
et s\'electionner ceux que vous voulez restaurer. Ce mode est similaire au 
programme de s\'election de fichier interactif standard d'Unix {\bf restore}.

Si vos fichiers ont \'et\'e \'elagu\'es, la commande {\bf restore} sera dans 
l'incapacit\'e de les trouver. Voyez ci-dessous pour plus de d\'etails sur ce cas 
de figure.

Dans la Console, apr\`es avoir saisi {\bf restore}, le menu suivant vous est 
pr\'esent\'e :

\footnotesize
\begin{verbatim}
First you select one or more JobIds that contain files
to be restored. You will be presented several methods
of specifying the JobIds. Then you will be allowed to
select which files from those JobIds are to be restored.
To select the JobIds, you have the following choices:
     1: List last 20 Jobs run
     2: List Jobs where a given File is saved
     3: Enter list of comma separated JobIds to select
     4: Enter SQL list command
     5: Select the most recent backup for a client
     6: Select backup for a client before a specified time
     7: Enter a list of files to restore
     8: Enter a list of files to restore before a specified time
     9: Find the JobIds of the most recent backup for a client
    10: Find the JobIds for a backup for a client before a specified time
    11: Enter a list of directories to restore for found JobIds
    12: Cancel
Select item:  (1-12):
\end{verbatim}
\normalsize

\begin{itemize}
\item Le choix 1 \'enum\`ere les 20 derniers jobs ex\'ecut\'es. Si vous trouvez 
   celui (ceux) que vous voulez, vous pouvez ensuite faire le choix 3 et entrer 
   son (leurs) JobId(s).

\item Le choix 2 affiche tous les Jobs ayant sauvegard\'e un fichier 
   sp\'ecifi\'e. Si vous trouvez celui (ceux) que vous voulez, vous pouvez ensuite 
   faire le choix 3 et entrer son (leurs) JobId(s).

\item Le choix 3 vous permet de saisir une liste de JobIds, s\'epar\'es par des 
   virgules. Les fichiers de ces jobs seront plac\'es dans l'arborescence afin 
   que vous puissiez s\'electionner ceux que vous voulez restaurer.

\item Le choix 4 vous permet d'entrer une requ\^ete SQL arbitraire. C'est 
   certainement le moyen le plus primitif pour trouver les jobs d\'esir\'es, 
   mais aussi le plus flexible. Si vous trouvez celui (ceux) que vous voulez, 
   vous pouvez ensuite faire le choix 3 et entrer son (leurs) JobId(s).
      
\item Le choix 5 s\'electionne automatiquement la full la plus r\'ecente, et toutes 
   les incr\'ementales et diff\'erentielles subs\'equentes \`a cette full pour un 
   client sp\'ecifi\'e. Il s'agit l\`a des jobs et fichiers qui, si vous les 
   restaurez, ram\`eneront votre syst\`eme \`a son dernier \'etat sauvegard\'e. 
   Les JobIds sont automatiquement charg\'es dans l'arborescence. C'est 
   probablement le plus pratique des choix propos\'es pour restaurer un 
   client \`a son \'etat le plus r\'ecent.

   Notez que ce processus de s\'election automatique ne s\'electionnera jamais 
   un job qui a \'echou\'e (termin\'e avec un statut d'erreur). Si vous disposez 
   d'un tel job dont vous voulez extraire des fichiers, vous devez 
   eplicitement entrer son JobId au niveau du choix 3 et choisir les fichiers 
   \`a restaurer.
   
   Si certains de jobs requis pour la restauration ont eu leurs enregistrements 
   de fichiers \'elagu\'es, la restauration sera incompl\`ete. Bacula ne d\'etecte 
   pas, pour l'instant, cette condition. Vous pouvez cependant la 
   contr\^oler en examinant attentivement la liste des jobs s\'electionn\'es 
   et affich\'es par Bacula. Si vous trouvez des jobs dont le champ JobFiles 
   est \`a z\'ero alors que ces fichiers auraient d\^u \^etre sauvegard\'es, alors 
   vous pouvez vous attendre \`a des probl\`emes.
   
   Si tous les enregistrements de fichiers ont \'et\'e \'elagu\'es, Bacula constatera 
   qu'il n'y a aucune r\'ef\'erence \`a aucun fichier pour le JobIds s\'electionn\'es 
   et vous en informera, et vous proposera de faire une restauration compl\`ete 
   (non s\'elective) de ces JobIds. Ceci est possible car Bacula sait encore 
   o\`u commencent les donn\'ees sur les volumes, m\^eme s'il ne sait plus o\`u sont 
   les fichiers individuellement.
   
\item Le choix 6 vous permet de sp\'ecifier une date et un heure. Bacula 
   s\'electionne alors automatiquement la plus r\'ecente full ant\'erieure \`a cette date 
   ainsi que les incr\'ementales et diff\'erentielles subs\'equentes \`a cette full et 
   ant\'erieures \`a cette date.

\item Le choix 7 vous permet de sp\'ecifier un ou plusieurs noms de fichiers 
   (le chemin absolu est requis) \`a restaurer. Les noms de fichiers sont saisis 
   un par un, \`a moins que vous ne pr\'ef\'eriez cr\'eer un fichier pr\'efix\'e du 
   caract\`ere "moins" (\lt{}) que Bacula consid\`ere comme une liste de fichier 
   \`a restaurer. Pour quitter ce mode, entrez une ligne vide.

\item Le choix 8 vous permet de sp\'ecifier une date et une heure avant 
   d'entrer les noms de fichiers. Voir le choix 7 pour plus de d\'etails.

\item Le choix 9 vous permet de d\'eterminer les JobIds de la sauvegarde 
   la plus r\'ecente pour un client. C'est essentiellement la m\^eme chose 
   que le choix 5 (le m\^eme code est utilis\'e), mais ces JobIds sont 
   conserv\'es en interne comme si vous les aviez saisis manuellement. 
   Vous pouvez alors faire le choix 11 pour restaurer un ou plusieurs 
   r\'epertoires.
   
\item Le choix 10 est le m\^eme que le 9, sauf qu'il vous permet d'entrer 
   une date butoir (comme pour le choix 6) pour la s\'election des JobIds. 
   Ces JobIds sont conserv\'es en interne comme si vous les aviez saisis manuellement.
   
\index[general]{Restaurer des r\'epertoires}
\item Le choix 11 vous permet d'entrer une liste de JobIds \`a partir de 
   laquelle vous pouvez s\'electionner les r\'epertoires \`a restaurer. La liste de 
   JobIds peut avoir \'et\'e \'etablie pr\'ec\'edemment \`a l'aide des choix 9 ou 10 
   du menu. Vous pouvez alors entrer le chemin absolu d'un r\'epertoire, ou 
   un nom de fichier pr\'efix\'e d'un signe "moins" (\lt{}) contenant la liste 
   des r\'epertoires \`a restaurer. Tous les fichiers des r\'epertoires s\'electionn\'es 
   seront restaur\'es, mais pas les sous-r\'epertoires, \`a moins que vous ne les 
   sp\'ecifiiez explicitement. 
   
\item  Le choix 12 vous permet d'abandonner la restauration.
\end{itemize}

A titre d'exemple, supposons que nous s\'electionnions l'option 5 (restaurer \`a 
l'\'etat le plus r\'ecent). Bacula vous demande alors le client d\'esir\'e ce qui, 
sur mon syst\`eme, se manifeste ainsi :

\footnotesize
\begin{verbatim}
Defined clients:
     1: Rufus
     2: Matou
     3: Polymatou
     4: Minimatou
     5: Minou
     6: MatouVerify
     7: PmatouVerify
     8: RufusVerify
     9: Watchdog
Select Client (File daemon) resource (1-9):
     
\end{verbatim}
\normalsize

Si vous n'avez qu'un client, il est automatiquement s\'electionn\'e. Dans le cas 
pr\'esent, j'entre {\bf Rufus} pour s\'electionner ce client. Bacula a 
maintenant conna\^itre le FileSet \`a restaurer, aussi il affiche :

\footnotesize
\begin{verbatim}
The defined FileSet resources are:
     1: Full Set
     2: Kerns Files
Select FileSet resource (1-2):
     
\end{verbatim}
\normalsize

J'opte pour le choix 1, ma sauvegarde full. En principe, vous n'aurez qu'un 
FileSet pour chaque job, et si vos machines de ressemblent (m\^emes syst\`emes), 
vous pouvez n'avoir qu'un seul FileSet pour tous vos clients.

A ce stade, Bacula d\'etient toutes les informations dont il a besoin pour 
trouver le jeu de sauvegardes le plus r\'ecent. Il va maintenant interroger le 
cataloguie, ce qui peut prendre un peu de temps, et afficher quelque chose 
comme :

\footnotesize
\begin{verbatim}
+-------+------+----------+-------------+-------------+------+-------+----------
--+
| JobId | Levl | JobFiles | StartTime   | VolumeName  | File | SesId |
VolSesTime |
+-------+------+----------+-------------+-------------+------+-------+----------
--+
| 1,792 | F    |  128,374 | 08-03 01:58 | DLT-19Jul02 |   67 |    18 |
1028042998 |
| 1,792 | F    |  128,374 | 08-03 01:58 | DLT-04Aug02 |    0 |    18 |
1028042998 |
| 1,797 | I    |      254 | 08-04 13:53 | DLT-04Aug02 |    5 |    23 |
1028042998 |
| 1,798 | I    |       15 | 08-05 01:05 | DLT-04Aug02 |    6 |    24 |
1028042998 |
+-------+------+----------+-------------+-------------+------+-------+----------
--+
You have selected the following JobId: 1792,1792,1797
Building directory tree for JobId 1792 ...
Building directory tree for JobId 1797 ...
Building directory tree for JobId 1798 ...
cwd is: /
$
\end{verbatim}
\normalsize

(Certaines colonnes sont tromqu\'ees pour des n\'ecessit\'es de mise en page).

Selon le nombre de {\bf JobFiles} pour chaque JobId, la construction de 
l'arborescence peut prendre un certain temps. Si vous constatez que tous les 
JobFiles sont \`a z\'ero, vos fichiers ont probalement \'et\'e \'elagu\'es et vous ne 
pourrez pas s\'electionner les fichiers individuellement : vous devrez 
restaurer tout ou rien.

Dans notre exemple, Bacula a trouv\'e quatre jobs qui comprennent la 
sauvegarde la plus r\'ecente du client et du FileSet sp\'ecifi\'es. Deux des jobs 
ont le m\^eme JobId car le job a \'ecrit sur deux volumes diff\'erents. Le 
troisi\`eme est une incr\'ementale qui n'a sauvegard\'e que 254 fichier sur les 
128 374 de la full. Le quatri\`eme est aussi une incr\'ementale, et n'a sauvegard\'e 
que 15 fichiers. 

Maintenant Bacula ins\`ere ces jobs dans l'arborescence, sans en marquer aucun 
pour restauration par d\'efaut. Il vous indique le nombre de fichiers dans 
l'arbre, et vous informe que le r\'epertoire de travail courant ({\bf cwd}) est 
/. Finalement, Bacula vous invite avec le signe (\$) \`a saisir des commandes 
pour vous d\'eplacer dans l'arborescence, et s\'electionner des fichiers.

Si vous voulez que tous les fichiers de l'arbre soient marqu\'es pour 
restauration \`a sa construction, tapez {\bf restore all}.

Plut\^ot que de choisir l'option 5 du premier menu (s\'electionner la 
sauvegarde la plus r\'ecente pour un client), si nous avions choisi l'option 3 
(Entrer une liste de JobIds \`a s\'electionner), et si nous avions saisi 
{\bf 1792,1797,1798}, nous serions arriv\'es au m\^eme point. 
             
Il faut noter un point si vous saisissez manuellement les JobIds : vous devez 
les entrer dans l'ordre o\`u ils ont \'et\'e ex\'ecut\'es (en g\'en\'eral, l'ordre croissant. 
Si vous les sasissez dans un ordre diff\'erent, vous courrez le risque de ne pas 
version la plus r\'ecente d'un fichier sauvegard\'e plusieurs fois si celui-ci a \'et\'e 
sauvegard\'e dans plusieurs jobs.

Entre vos JobIds directement peut aussi vous permettre de restaurer depuis 
un job qui a \'ecrit des donn\'ees sur les volumes mais qui s'est termin\'e en erreur.

Dans le mode s\'election de fichiers, vous pouvez utiliser {\bf help} ou une 
question (?) pour produire un r\'esum\'e des commandes disponibles :

\footnotesize
\begin{verbatim}
 Command    Description
  =======    ===========
  cd         change current directory
  count      count marked files in and below the cd
  dir        long list current directory, wildcards allowed
  done       leave file selection mode
  estimate   estimate restore size
  exit       same as done command
  find       find files, wildcards allowed
  help       print help
  ls         list current directory, wildcards allowed
  lsmark     list the marked files in and below the cd
  mark       mark dir/file to be restored recursively in dirs
  markdir    mark directory name to be restored (no files)
  pwd        print current working directory
  unmark     unmark dir/file to be restored recursively in dir
  unmarkdir  unmark directory name only no recursion
  quit       quit and do not do restore
  ?          print help
\end{verbatim}
\normalsize

Par d\'efaut, aucun fichier n'est s\'electionn\'e pour restauration (sauf si vous  
avez ajout\'e {\bf all} \`a la ligne de commande). Si, \`a ce stade, vous voulez 
tout restaurer, vous devriez saisir {\bf mark *}, puis {\bf done}, Bacula 
\'ecrira alors les donn\'ees bootstrap dans un fichier et sollicitera votre 
approbation pour d\'emarrer la restauration.

Si vous n'utilisez pas {\bf mark *}, vous commencez avec une s\'election vide. 
Vous pouvez simplement regarder et marquer ({\bf mark}) les fichiers et/ou 
r\'epertoires qui vous int\'eressent. Il est ais\'e de commettre une erreur dans ces 
op\'erations, et la gestion des erreurs dans Bacula n'est pas parfaite, aussi 
contr\^olez votre travail avec la commande {\bf ls} ou {\bf dir} pour voir 
quels fichiers ont \'et\'e s\'electionn\'es. Les fichiers s\'electionn\'es sont pr\'ec\'ed\'es 
d'une ast\'erisque.

Pour contr\^oler ce qui est marqu\'e et ce qui ne l'est pas utilisez la commande 
{\bf count} qui affiche :

\footnotesize
\begin{verbatim}
128401 total files. 128401 marked to be restored.
     
\end{verbatim}
\normalsize

Chacune des commandes ci-dessus sera expliqu\'e plus en d\'etail dans la 
prochaine section. Poursuivons avec notre exemple, en validant la restauration de 
tous les fichiers. En saisissant {\bf done}, Bacula affiche :

\footnotesize
\begin{verbatim}
Bootstrap records written to /home/kern/bacula/working/restore.bsr
The restore job will require the following Volumes:
   
   DLT-19Jul02
   DLT-04Aug02
128401 files selected to restore.
Run Restore job
JobName:    kernsrestore
Bootstrap:  /home/kern/bacula/working/restore.bsr
Where:      /tmp/bacula-restores
Replace:    always
FileSet:    Kerns Files
Client:     Rufus
Storage:    SDT-10000
JobId:      *None*
OK to run? (yes/mod/no):
    
\end{verbatim}
\normalsize

Examinez chaque \'el\'ement attentivement pour vous assurer que tout est 
conforme \`a ce que vous souhaitez. En particulier, v\'erifiez la ligne {\bf where}, 
qui vous indique dans quelle partie du syst\`eme de fichiers vos donn\'ees 
seront restaur\'ees, et quel client va les recevoir (par d\'efaut, les 
restaurations ont lieu sur le client d'origine). Ces param\`etres n'auront pas 
forc\'ement les bonnes valeurs, mais vous pouvez les modifier \`a l'aide 
de la commande {\bf mod} et en vous laissant guider par l'invite de la 
console.

L'affichage ci-dessus suppose que vous ayez d\'efini une ressource Job de type 
{\bf restore} dans le fichier de configuration de votre Director. en 
principe, vous n'en n'aurez besoin que d'une, car, par nature, une 
restauration est une op\'eration essentiellement manuelle. A l'aide de la 
Console, vous pourrez modifier le job Restore pour faire ce que vous voulez 
qu'il fasse.

Un exemple de ressource Job de type restore est donn\'e plus bas.

Pour en revenir \`a notre exemple, en plus de v\'erifier le client, il est sage 
de v\'erifier que le p\'eriph\'erique de stockage choisi par Bacula est le bon. 
Bien que le FileSet soit pr\'esent\'e, il est en fait ignor\'e dans la restauration. 
Le processus de restauration choisit ses fichiers en lisant le fichier 
{\bf bootstrap}, et restaure tous les fichiers associ\'es au JobId consid\'er\'e 
si ce fichier n'est pas sp\'ecifi\'e. 

Enfin, avant de lancer la restauration, notez que le lieu par d\'efaut pour les 
fichiers restaur\'es n'est pas leur emplacement d'origine mais le r\'epertoire 
{\bf /tmp/bacula-restores}. Vous pouvez modifier cette valeur par d\'efaut dans 
le fichier de configuration du Director, ou avec l'option {\bf mod}. Si vous 
voulez restaurer les fichiers \`a leurs emplacements d'origine, modifiez l'option 
{\bf where} : sp\'ecifiez la racine ({\bf /} ou rien du tout).

Si vous entrez maintenant {\bf yes}, Bacula lance la restauration. le Storage 
Daemon va d'abord requ\'erir le volume {\bf DLT-19Jul02}, puis le {\bf DLT-04Aug02} 
une fois qu'il aura extrait les fichiers requis du premier.

\section{S\'electionner des fichiers par leurs noms}
\index[general]{S\'electionner des fichiers par leurs noms}
\index[general]{Noms de fichiers!S\'electionner des fichiers par leurs}
\addcontentsline{toc}{section}{S\'electionner des fichiers par leurs noms}

Si vous n'avez qu'un petit nombre de fichiers \`a restaurer dont vous connaissez 
les noms, vous pouvez, aux choix, placer ces noms dans un fichier qui sera 
lu par Bacula, ou saisir les noms un par un. Les noms de fichier doivent inclure 
le chemin absolu. Les caract\`eres jokers ne peuvent \^etre utilis\'es.

Pour saisir la liste, choisissez l'option 7 dans le menu de la commande {\bf restore} :

\footnotesize
\begin{verbatim}
To select the JobIds, you have the following choices:
     1: List last 20 Jobs run
     2: List Jobs where a given File is saved
     3: Enter list of comma separated JobIds to select
     4: Enter SQL list command
     5: Select the most recent backup for a client
     6: Select backup for a client before a specified time
     7: Enter a list of files to restore
     8: Enter a list of files to restore before a specified time
     9: Find the JobIds of the most recent backup for a client
    10: Find the JobIds for a backup for a client before a specified time
    11: Enter a list of directories to restore for found JobIds
    12: Cancel
Select item:  (1-12):
\end{verbatim}
\normalsize

Vous \^etes alors invit\'e \`a pr\'eciser le client :

\footnotesize
\begin{verbatim}
Defined Clients:
     1: Timmy
     2: Tibs
     3: Rufus
Select the Client (1-3): 3
\end{verbatim}
\normalsize

Si vous n'avez qu'un client, il est s\'electionn\'e automatiquement.
Finalement, Bacula vous demande d'entrer un nom de fichier :

\footnotesize
\begin{verbatim}
Enter filename:
\end{verbatim}
\normalsize

Vous pouvez, \`a ce stade, saisir le chemin absolu et le nom du fichier :

\footnotesize
\begin{verbatim}
Enter filename: /home/kern/bacula/k/Makefile.in
Enter filename:
\end{verbatim}
\normalsize

Si Bacula ne peut en trouver aucune copie, il affiche ce qui suit :

\footnotesize
\begin{verbatim}
Enter filename: junk filename
No database record found for: junk filename
Enter filename:
\end{verbatim}
\normalsize

Si vous souhaitez que Bacula r\'ecup\`ere la liste des fichiers \`a restaurer depuis 
un fichier, r\'edigez ce fichier et donnez lui un nom commen\c{c}ant par le signe 
moins (\lt{}) et saisissez-le ici. Lorsque vous avez entr\'e tous les noms de 
fichiers, validez une ligne vide. Bacula \'ecrit maintenant le fichier 
bootstrap, vous indique les cartouches qui seront utilis\'ees, et vous propose 
de valider la restauration :

\footnotesize
\begin{verbatim}
Enter filename:
Automatically selected Storage: DDS-4
Bootstrap records written to /home/kern/bacula/working/restore.bsr
The restore job will require the following Volumes:
   
   test1
1 file selected to restore.
Run Restore job
JobName:    kernsrestore
Bootstrap:  /home/kern/bacula/working/restore.bsr
Where:      /tmp/bacula-restores
Replace:    always
FileSet:    Kerns Files
Client:     Rufus
Storage:    DDS-4
When:       2003-09-11 10:20:53
Priority:   10
OK to run? (yes/mod/no):
\end{verbatim}
\normalsize

Il est possible d'automatiser la s\'election des fichiers en pla\c{c}ant votre liste 
de fichiers dans, part exemple, {\bf /tmp/file-list}, puis en utilisant la 
commande suivante :

\footnotesize
\begin{verbatim}
restore client=Rufus file=</tmp/file-list
\end{verbatim}
\normalsize

Si, en modifiant les param\`etres du job restauration, vous constatez que Bacula 
vous demande d'entrer un num\'ero de job, c'est vous n'avez pour l'instant sp\'ecifi\'e 
ni num\'ero de job, ni fichier bootstrap. Entrez simplement z\'ero pour pouvoir 
continuer et s\'electionner une autre option \`a modifier.

\label{CommandArguments}

\section{Arguments de la ligne de commande}
\index[general]{Arguments!ligne de commande}
\index[general]{Arguments de la ligne de commande}
\addcontentsline{toc}{section}{Arguments de la ligne de commande}

Si tout ce qui pr\'ec\`ede vous a sembl\'e compliqu\'e, vous admettrez certainement 
que ce n'est vraiment pas le cas apr\`es quelques essais. Il est possible de 
faire tout ce qui vient d'\^etre vu en utilisant la ligne de commande, \`a 
l'exception de la s\'election du FileSet. Voici une telle ligne de commande :

\footnotesize
\begin{verbatim}
restore client=Rufus select current all done yes
\end{verbatim}
\normalsize

Le sp\'ecification {\bf client=Rufus} s\'electionne automatiquement le client Rufus, 
l'option {\bf current} pr\'ecise que vous voulez une restauration \`a l'\'etat le plus 
r\'ecent possible, et le  {\bf yes} \'elude l'invite finale {\bf yes/mod/no} et 
ex\'ecute directement la restauration.

Voici la liste des arguments de la ligne de commandes :

\begin{itemize}
\item {\bf all} -- s\'electionne tous les fichiers pour la restauration.
\item {\bf select} -- utilise la s\'election via l'arborescence.
\item {\bf done} -- permet de quitter le mode de s\'election dans l'arborescence.
\item {\bf current} -- s\'electionne automatiquement le jeu de sauvegardes le plus 
   r\'ecent pour le client sp\'ecifi\'e.
\item {\bf client=xxxx} -- S\'electionne le client sp\'ecifi\'e.  
\item {\bf jobid=nnn} -- Sp\'ecifie un JobId ou une liste de JobIds s\'epar\'es par des 
   virgules pour la restauration.
\item {\bf before=YYYY-MM-DD HH:MM:SS} -- Sp\'ecifie une date et un horaire. Bacula 
   s\'electionne le plus r\'ecent des jeux de sauvegardes ant\'erieurs \`a la date sp\'ecifi\'ee.
   Cette commande n'est pas tr\`es conviviale, en effet, vous devez sp\'ecifier la date 
   et l'heure en respectant exactement le mod\`ele.
\item {\bf file=filename} -- Sp\'ecifie un nom de fichier \`a restaurer. Vous devez 
   sp\'ecifier le chemin absolu vers le fichier. Si vous pr\'efixez l'entr\'ee d'un signe moins 
   (\lt{}), Bacula consid\`ere que ce fichier existe et qu'il contient la liste des 
   fichiers \`a restaurer. Les sp\'ecifications multiples {\bf file=xxx} peuvent \^etre 
   utilis\'ees en ligne de commandes.
\item {\bf jobid=nnn} -- Sp\'ecifie un JobId \`a restaurer. 
\item {\bf pool=pool-name} -- Sp\'ecifie un nom de pool \`a utiliser pour la s\'election des 
   volumes au niveau des options 5 et 6 (restauration \`a l'\'etat le plus r\'ecent et 
   restauration \`a l'\'etat le plus r\'ecent avant une date donn\'ee). Ceci vous permet d'avoir 
   plusieurs pools, dont un \'eventuellement hors site et l'autre sur place disponible pour 
   les restaurations.
\item {\bf yes} -- Ex\'ecute automatiquement la restauration sans passer par l'invite finale 
   de validation/modification surtout utile pour l'utilisation dans des scripts). 
   \end{itemize}

\section{Restaurer les attributs de fichiers}
\index[general]{Attributs de fichiers!Restaurer}
\index[general]{Restaurer les attributs de fichiers}
\addcontentsline{toc}{section}{Restaurer les attributs de fichiers}

Selon la fa\c{c}on dont vous restaurez, vous pouvez ou non restaurer les 
attributs de fichiers \`a leur \'etat initial. Voici quelques uns des 
probl\`emes auxquels vous pouvez \^etre confront\'es, et, pour les 
restaurations sur la machine d'origine, comment les \'eviter.

\begin{itemize}
\item Vous avez sauvegard\'e des fichiers sur une machine, et les restaurez
   sur une autre qui a peut-\^etre un autre syst\`eme d'exploitation ou des 
   utilisateurs/groupes diff\'erents. Bacula fait du mieux qu'il peut dans ces 
   situations. Notez qu'utilisateurs et groupes sont sauvegard\'es au format 
   num\'erique, et qu'ils peuvent donc se r\'ef\'erer \`a d'autres utilisateurs et 
   groupes sur un autre syst\`eme.
\item Vous restaurez dans un r\'epertoire existant sur lequel portent des 
   restrictions du droit de cr\'eation. Bacula tente alors de tout r\'etablir, 
   mais sans parcourir la cha\^ine compl\`ete des r\'epertoires ni les modifier 
   durant la restauration. En fait, ce que pourra faire Bacula pour r\'etablir 
   les permissions correctement d\'epend pour beaucoup de votre syst\`eme 
   d'exploitation.
\item Vous faites une restauration recursive d'une arborescence. Dans ce cas, 
   de figure, Bacula restaure un fichier avant de restaurer l'entr\'ee de son 
   r\'epertoire parent. Dans le processus de restauration du fichier, Bacula 
   cr\'ee le r\'epertoire parent avec des permissions ouvertes et le m\^eme 
   propri\'etaire que le fichier restaur\'e. Alors, lorsque Bacula tente de restaurer 
   le r\'epertoire en lui m\^eme, il se rend compte qu'il existe d\'ej\`a (situation 
   similaire \`a la pr\'ec\'edente). Si vous avez fix\'e l'option "Replace" \`a "never" 
   lors du lancement du job, alors Bacula ne modifie pas les permissions et 
   propri\'et\'es du r\'epertoire pour s'accorder \`a ce qu'elles \'etaient lors de la 
   sauvegarde. Vous devriez aussi noter une divergence entre le nombre de fichiers 
   effectivement restaur\'es et le nombre de fichiers attendus. Si vous voulez 
   \'eviter ces inconv\'enients, fixez l'option "Replace" \`a "always", ainsi 
   Bacula sera en mesure de modifier les propri\'etaire et permissions des 
   r\'epertoires pour les ramener \`a leurs \'etats d'origine. Le nombre de 
   fichiers restaur\'es devrait cette fois \^etre identique \`a celui attendu.

\item Vous avez s\'electionn\'e un ou plusieurs fichiers d'un r\'epertoire sans 
   s\'electionner le r\'epertoire lui-m\^eme. Dans ce cas, si le r\'epertoire 
   n'existe pas d\'ej\`a, Bacula le cr\'ee avec des attributs par d\'efaut qui ne 
   seront peut-\^etre pas ceux d'origine. Si vous ne voulez pas s\'electionner 
   un r\'epertoire et tout son contenu, mais seulement quelques objets dans 
   ce r\'epertoire en les marquant individuellement, vous devriez utiliser 
   la commande {\bf markdir} pour s\'electionner un r\'epertoire de plus haut 
   niveau (un \`a la fois) si vous voulez que les entr\'ees de r\'epertoires 
   soient restaur\'ees correctement.
\end{itemize}

\label{Windows}

\section{Restaurer sur Windows}
\index[general]{Restaurer sur Windows}
\index[general]{Windows!Restaurer sur}
\addcontentsline{toc}{section}{Restaurer sur Windows}
Sur les syst\`emes WinNT/2K/XP, Bacula restaure les fichiers avec les droits 
et permissions d'origine comme on s'y attend. Ceci est aussi v\'erifi\'e si vous 
restaurez ces fichiers vers un autre r\'epertoire (avec l'option "where") que celui 
d'origine. Cependant, si le nouveau r\'epertoire n'existe pas, le File Daemon 
tente de le cr\'eer. Dans certains cas, il n'y parvient pas. S'il y parvient, le 
r\'epertoire cr\'e\'e appartient \`a l'utilisateur qui ex\'ecute le File Daemon, c'est-\`a-dire 
SYSTEM. Dans ce cas, il se peut que vous ayez des difficult\'es pour acc\'eder aux 
fichiers fraichement restaur\'es.

Pour \'eviter ce probl\`eme, vous devriez cr\'eer le r\'epertoire alternatif avant 
de lancer la restauration. Bacula ne changera pas les attributs de ce r\'epertoire, 
du moment que ce n'est pas l'un des r\'epertoires \`a restaurer.

Le r\'epertoire de restauration par d\'efaut est {\bf /tmp/bacula-restores/}, qui devient  
{\bf /tmp/bacula-restores/e/} si vous restaurez depuis le disque {\bf E}. 
Aussi, assurez-vous que ce r\'epertoire existe avant de lancer la restauration, ou 
utilisez l'option {\bf mod} pour s\'electionner un r\'epertoire destination existant.

Certains utilisateurs ont signal\'e des probl\`emes en restaurant des fichiers 
qui participent \`a Active Directory. Ils ont aussi rapport\'e que le changement de 
l'Id utilisateur sous lequel est ex\'ecut\'e Bacula de SYSTEM en un Id d'administrateur 
du domaine r\'esout le probl\`eme.

\section{Une restauration peut prendre du temps}
\index[general]{temps!Restauration}
\index[general]{Une restauration peut prendre du temps}
\addcontentsline{toc}{section}{Une restauration peut prendre du temps}

Restaurer des fichiers est g\'en\'eralement {\bf beaucoup} plus lent que de les 
sauvegarder, ce pour plusieurs raisons. La premi\`ere est que lors d'une sauvegarde, 
la cartouche est normalement d\'ej\`a positionn\'ee, Bacula n'a qu'\`a \'ecrire dessus. 
D'autre part, les restaurations \'etant si rares (par rapport aux sauvegardes), 
Bacula ne garde dans le catalogue que l'emplacement sur la cartouche du premier 
fichier et du premier bloc pour chaque job, et non l'emplacement de chaque fichier, 
ce qui occuperait trop de place dans le catalogue.

Bacula se place d'abord sur la bonne marque de fichier sur la cartouche, puis 
sur le bloc correct, puis lit s\'equentiellement chaque enregistrement jusqu'\`a 
trouver ceux correspondant aux fichier que vous voulez restaurer. Une fois ces 
fichiers restaur\'es, Bacula cesse de lire la cartouche.

Enfin, au lieu de simplement lire un fichier comme pour une sauvegarde, Bacula 
doit, lors d'une restauration, cr\'eer les fichiers, tandis que le syst\`eme 
d'exploitation doit, de son cot\'e, allouer de l'espace disque pour ces fichiers 
restaur\'es.

Pour toutes ces raisons, le processus de restauration est g\'en\'eralement beaucoup 
plus lent que celui de sauvegarde (une restauration peut prendre trois fois 
plus de temps que la sauvegarde).

\section{Probl\`emes lors de la restauration de fichiers}
\index[general]{Fichiers!Probl\`emes de restauration}
\index[general]{Probl\`emes lors de la restauration de fichiers}
\addcontentsline{toc}{section}{Probl\`emes lors de la restauration de fichiers}

Les probl\`emes que les utilisateurs rencontrent le plus souvent lors des restaurations 
sont des messages d'erreurs tels que :

\footnotesize
\begin{verbatim}
04-Jan 00:33 z217-sd: RestoreFiles.2005-01-04_00.31.04 Error:
block.c:868 Volume data error at 20:0! Short block of 512 bytes on
device /dev/tape discarded.
\end{verbatim}
\normalsize

ou 

\footnotesize
\begin{verbatim}
04-Jan 00:33 z217-sd: RestoreFiles.2005-01-04_00.31.04 Error:
block.c:264 Volume data error at 20:0! Wanted ID: "BB02", got ".".
Buffer discarded.
\end{verbatim}
\normalsize

Ces deux types de messages indiquent que vous avez probablement utilis\'e votre 
lecteur en mode "blocs de taille fixe" plut\^ot qu'en mode "blocs de taille 
variable". Le mode "blocs de taille fixe" fonctionne avec tout programme 
qui lit les cartouches s\'equentiellement tel que {\bf tar}, cependant Bacula 
repositionne la bande suivant les blocs lors des restaurations, ce qui lui 
permet d'am\'eliorer les performances en restauration de plusieurs ordres 
de grandeur lorsqu'il s'agit de restaurer quelques fichiers isol\'es. Il existe 
plusieurs moyens pour vous tirer de ce mauvais pas.

Tentez-les l'un apr\`es l'autre, en r\'etablissant votre ressource Device apr\`es 
chacun des tests :

\begin{enumerate}
\item D\'esactivez le positionnement par blocs ("Block Positioning = no" 
   dans la ressource Device) et essayez de restaurer. Cette directive est 
   r\'ecente et n'est pas encore bien test\'ee.
\item R\'eglez \`a 512 les tailles minimum et maximum de blocs  
   ("Minimum Block Size = 512" et "Maximum  Block Size = 512") et essayez de 
   restaurer. Si vous \^etes en mesure de d\'eterminer la taille de blocs 
   utilis\'ee par votre lecteur pour \'ecrire les donn\'ees, vous devriez essayer 
   cette valeur si la restauration a \'echou\'e avec 512. 
\item Editez le fichier restore.bsr \`a l'invite yes/mod/no de la 
   commande Run xxx avant de valider la restauration, et supprimez tous les 
   enregistrements VolBlock. Ce sont eux qui causent les repositionnements de 
   la bande et les probl\`emes qui s'ensuivent si vous utilisez des blocs de taille 
   fixe sur votre lecteur. Les commandes VolFile provoquent aussi le 
   repositionnement, mais celui-ci fonctionne ind\'ependamment de la taille des blocs.
\item Utilisez bextract pour extraire vos fichiers -- ce programme lit les 
   volumes s\'equentiellement si vous utilisez la fonctionnalit\'e des listes 
   d'inclusions, ou si vous utilisez un fichier .bsr (priv\'e des enregistrements 
   VolBlock) \`a l'invite "yes/mod/no" qui pr\'ec\`ede le lancement de la 
   restauration.
\end{enumerate}

\section{Erreurs de restaurations}
\index[general]{Erreurs!Restauration}
\index[general]{Erreurs de restauration}
\addcontentsline{toc}{section}{Erreurs de restauration}

Il existe de multiples causes qui peuvent \^etre \`a l'origine de messages d'alerte ou d'erreurs 
lors des restaurations. Les plus courantes sont les suivantes :

\begin{description}

\item [file count mismatch]
   Cette erreur peut se produire dans les cas suivants : 
   \begin{itemize}
   \item Vous avez enjoint Bacula \`a ne pas \'ecraser les fichiers 
      existants ou plus r\'ecents.
   \item Bacula a commis une erreur dans le d\'ecompte des fichiers/r\'epertoires. 
      Ceci est un probl\`eme inh\'erent \`a la complexit\'e des r\'epertoires, liens 
      symboliques/durs, etc. Contr\^olez simplement que tous les fichiers voulus ont \'et\'e 
      effectivement restaur\'es.
   \end{itemize}
\item [file size error]
   Lorsque Bacula restaure un fichier, il v\'erifie que la taille du fichier 
   restaur\'e est conforme aux donn\'ees d'\'etat enregistr\'ees au d\'ebut de la sauvegarde 
   du fichier. En cas de divergence, Bacula affiche un message d'erreur. Cet 
   divergence survient presque toujours \`a cause d'une \'ecriture du fichier 
   au cours de sa sauvegarde. Dans ce cas, la taille du fichier restaur\'e 
   est plus grande que la taille enregistr\'ee dans les donn\'ees d'\'etat. Cette 
   erreur se produit souvent avec les fichiers de journaux.

   Si en revanche la taille du fichier restaur\'ee est plus petite, vous devriez 
   pousser vos investigations du cot\'e d'un \'eventuel probl\`eme de bande et 
   contr\^oler les rapports de Bacula ainsi que votre syst\`eme de journalisation.
\end{description}

\section{Un exemple de ressource Restore Job}
\index[general]{Exemple ressource Restore Job }
\index[general]{Ressource!Exemple Restore Job }
\addcontentsline{toc}{section}{Exemple ressource Restore Job}

\footnotesize
\begin{verbatim}
Job {
  Name = "RestoreFiles"
  Type = Restore
  Client = Any-client
  FileSet = "Any-FileSet"
  Storage = Any-storage
  Where = /tmp/bacula-restores
  Messages = Standard
  Pool = Default
}
\end{verbatim}
\normalsize
Si la directive {\bf Where} n'est pas pr\'ecis\'ee, les fichiers sont restaur\'es \`a 
leur emplacement d'origine. 
\label{Selection}

\section{Les commande de s\'election de fichiers}
\index[general]{Commandes!s\'election fichiers}
\index[general]{Fichiers S\'election Commandes }
\addcontentsline{toc}{section}{Commandes de s\'election de fichiers}

Une fois que vous avez s\'electionn\'e les jobs \`a restaurer et apr\`es que Bacula a cr\'e\'e 
l'arborescence des r\'epertoires en m\'emoire, vous entrez dans le mode de s\'election 
de fichiers, ce qui est rappel\'e par l'invite ({\bf \$}). Dans ce mode, vous 
pouvez utiliser les commandes \'enum\'er\'ees plus haut. Vous pouvez naviguer dans 
l'arborescence avec la commande {\bf cd} tout comme vous le feriez dans  
un syst\`eme de fichiers. Lorsque vous \^etes dans un r\'epertoire, vous pouvez en 
s\'electionner des fichiers ou r\'epertoires pour restauration. Par d\'efaut, aucun 
fichier n'est s\'electionn\'e. Si vous souhaitez au contraire partir d'une situation 
o\`u tous les fichiers sont s\'electionn\'es, saisissez simplement : {\bf cd /} et 
{\bf mark *}. Sinon, s\'electionnez vos fichiers avec la commande {\bf mark}. Les 
commandes disponibles sont :

\begin{description}

\item [cd]
   La commande {\bf cd} change le r\'epertoire courant en l'argument sp\'ecifi\'e. les 
   caract\`eres joker ne sont pas admis. 

   Notez que sur les syst\`emes Windows, les diff\'erents disques (c:, d:, ...) sont 
   trait\'es comme des r\'epertoires dans l'arborescence. Une cons\'equence est que vous 
   devez saisir {\bf cd c:} ou \'eventuellement {\bf cd C:} pour descendre dans le 
   premier r\'epertoire.

\item [dir]
   \index[dir]{dir }
   La commande {\bf dir} est similaire \`a la commande {\bf ls}, mais produit un 
   affichage au format long (tous les d\'etails). Cette commande peut \^etre un peu 
   plus lente que la commande {\bf ls} car elle n\'ecessite l'acc\`es au catalogue 
   pour produire les informations d\'etaill\'ees concernant chaque fichier.

\item [estimate]
   \index[dir]{estimate }
   La commande {\bf estimate} affiche un aper\c{c}u du nombre total de fichiers dans 
   l'arbre, de ceux s\'electionn\'es pour restauration, et une estimation du nombre 
   d'octets \`a restaurer. Ceci peut-\^etre tr\`es utile si la machine o\`u vous comptez 
   restaurer est limit\'ee en espace disque.

\item [find]
   \index[dir]{find }
   La commande {\bf find} prend un ou plusieurs arguments et affiche tous les fichiers 
   de l'arbre qui satisfont ces arguments. Les arguments peuvent comporter 
   des caract\`eres joker. Cette commande est similaire \`a la commande Unix 
   {\bf find / -name arg}.

\item [ls]
   La commande {\bf ls} produit la liste des fichiers du r\'epertoire courant, comme 
   le ferait la commande Unix {\bf ls}. Vous pouvez sp\'ecifier un argument 
   comportant des caract\`eres joker, auquel cas seuls les fichiers concern\'es seront 
   list\'es. Le nom de tout fichier s\'electionn\'e pour restauration est pr\'ec\'ed\'e 
   d'un ast\'erisque ({\bf *}). Les noms de r\'epertoires sont suivis d'une barre oblique 
   avant (slash {\bf /}) pour les distinguer des noms de fichiers.

\item [lsmark]
   \index[fd]{lsmark}
   La commande {\bf lsmark} est similaire \`a la commande {\bf ls}, mais de port\'ee  
   restreinte aux fichiers s\'electionn\'es pour restauration. D'autre part, contrairement 
   \`a {\bf lsmark}, elle descend r\'ecursivement dans les sous-r\'epertoire du 
   r\'epertoire s\'electionn\'e.

\item [mark]
   \index[dir]{mark }
   La commande {\bf mark} vous permet de s\'electionner vos fichiers pour restauration. 
   Elle prend pour unique argument le nom du fichier ou r\'epertoire (du r\'epertoire 
   courant) \`a restaurer. L'argument peut comporter des caract\`eres joker, auquel cas 
   tous les fichiers qui co\"incident avec le motif sp\'ecifi\'e sont s\'electionn\'es pour 
   restauration. Si un r\'epertoire du r\'epertoire courant co\"incide avec l'argument, 
   alors ce r\'epertoire et tous les fichiers qu'il contient sont r\'ecursivement 
   s\'electionn\'es pour restauration. Chacun des fichiers s\'electionn\'e est identifiable par 
   l'ast\'erisque qui pr\'ec\`ede son nom dans la liste produite par les commandes {\bf ls} 
   ou {\bf dir}. Notez que fournir un chemin absolu en argument de la commande 
   {\bf mark} ne produit pas le r\'esultat qu'on pourrait en attendre pour s\'electionner 
   un fichier ou un r\'epertoire du r\'epertoire courant. La commande {\bf mark} travaille 
   sur le r\'epertoire courant et sur les r\'epertoires enfants, mais ne remonte pas vers les 
   r\'epertoires de plus haut niveau.

   Apr\`es ex\'ecution, la commande {\bf mark} affiche un bref r\'esum\'e :

\footnotesize
\begin{verbatim}
    No files marked.    

\end{verbatim}
\normalsize

   si aucun fichier n'a \'et\'e s\'electionn\'e, ou :

\footnotesize
\begin{verbatim}
    nn files marked.
    
\end{verbatim}
\normalsize

   si certains fichiers ont \'et\'e s\'electionn\'es.

\item [unmark]
   \index[dir]{unmark }
   la commande {\bf unmark} fonctionne exactement comme la commande {\bf mark}, mais 
   sert \`a d\'es\'electionner les fichiers sp\'ecifi\'es pour qu'ils ne soient pas 
   restaur\'es. Notez que la commande {\bf unmark} travaille sur le r\'epertoire courant 
   et sur les r\'epertoires enfants, mais ne remonte pas vers les r\'epertoires de plus 
   haut niveau. Si vous souhaitez d\'es\'electionner l'ensembles des fichiers, placez-vous  
   \`a la racine ({\bf cd /}) avant de saisir {\bf unmark *}.

\item [pwd]
   \index[dir]{pwd }
   La commande {\bf pwd} affiche le r\'epertoire courant. Elle ne prend aucun argument.

\item [count]
   \index[dir]{count }
   La commande {\bf count} affiche le nombre total de fichiers dans l'arborescence des 
   fichiers et le nombre de fichiers s\'electionn\'es pour restauration.

\item [done]
   \index[dir]{done }
   Cette commande permet de quitter le mode de s\'election de fichiers.

\item [exit]
   \index[fd]{exit }
   Cette commande permet de quitter le mode de s\'election de fichiers (identique \`a {\bf done}).

\item [quit]
   \index[fd]{quit }
   Cette commande permet de quitter le mode de s\'election de fichiers sans ex\'ecuter la restauration.

\item [help]
   \index[fd]{help }
   Cette commande affiche un aper\c{c}u des commandes disponibles.

\item [?]
   Cette commande est identique \`a {\bf help}.
\end{description}

\label{database_restore}
\section{Restaurer lorsque tout va de travers}
\index[general]{Restaurer lorsque tout va de travers}
\index[general]{Restaurer le catalogue}
\index[general]{Catalogue!Restaurer}
\addcontentsline{toc}{section}{Restaurer lorsque tout va de travers}

Je pr\'esente ici quelques-uns des probl\`emes qui peuvent survenir et 
rendre les op\'erations de restaurations plus difficiles, et quelques id\'ees 
pour se sortir de ces situations d\'elicates. Des informations plus 
sp\'ecifiques aux restaurations compl\`etes des \ilink{clients}{restore_client}
ou du \ilink{serveur}{restore_server} sont fournies dans le chapitre 
\ilink{Disaster recovery avec Bacula}{_ChapterRescue}.

\begin{description}
\item[Probl\`eme]
   Mon catalogue est corrompu.
\item[Solution]
   Pour SQLite, utilisez la commande {\bf vacuum} pour tenter de r\'ecup\'erer 
   le catalogue. Pour MySQL ou PostgreSQL, consultez la documentation 
   officielle de la base de donn\'ees. Des outils sp\'ecifiques vous permettront 
   de contr\^oler et r\'eparer votre catalogue.

   Dans le cas contraire, vous devrez restaurer le catalogue.
\item[Probl\`eme]
   Comment restaurer mon catalogue ?
\item[Solution]
   Si vous avez sauvegard\'e votre catalogue quotidiennement (comme il se doit...) 
   et avez fait un fichier {\it bootstrap}, vous pouvez imm\'ediatement 
   recharger votre base de donn\'ees. Faites une copie de votre base courante, 
   puis r\'einitialisez-la avec les scripts suivants :
\begin{verbatim}
   ./drop_bacula_tables
   ./make_bacula_tables
\end{verbatim}
   Apr\`es r\'einitialisation de la base, vous devriez pouvoir d\'emarrer Bacula.
   Si vous tentez maintenat d'utiliser la commande {\bf run}, \c{c}a ne 
   marchera pas puisque le catalogue est vierge. Cependant, vous pouvez 
   ex\'ecuter manuellement un job et sp\'ecifier votre fichier bootstrap.
   Pour cela, utilisez la commande {\bf run} et choisissez un job de type 
   {\bf retore}. Si vous utilisez le bacula-dir.conf par d\'efaut, il s'agit 
   de {\bf RestoreFiles}. Vous devriez obtenir quelque chose comme :

\footnotesize
\begin{verbatim}
Run Restore job
JobName:    RestoreFiles
Bootstrap:  /home/kern/bacula/working/restore.bsr
Where:      /tmp/bacula-restores
Replace:    always
FileSet:    Full Set
Client:     rufus-fd
Storage:    File
When:       2005-07-10 17:33:40
Catalog:    MyCatalog
Priority:   10
OK to run? (yes/mod/no): 
\end{verbatim}
\normalsize
   Plusieurs param\`etres seront diff\'erents dans votre cas. Vous aller modifier 
   (commande {\bf mod}) le param\`etre {\bf Bootstrap} pour qu'il pointe 
   vers votre fichier bootstrap, et vous assurer que les autres param\`etres 
   sont corrects. Notez que le FileSet est ignor\'e lorque vous utilisez un 
   fichier bootstrap. Une fois que vous avez fix\'e tous les bons param\`etres, 
   ex\'evutez le job, vous devriez r\'ecup\'erer la sauvegarde de votre catalogue. 
   Il vous reste alors \`a r\'eg\'en\'erer votre base de donn\'ees \`a partir du 
   fichier de sauvegarde ASCII.

\item[Solution]
   Si vous avez sauvegard\'e votre catalogue mais n'avez pas fait de fichier 
   bootstrap, la reconstruction du catalogue sera un peu plus difficile. 
   Il vous faudra probablement utiliser {\bf bextract} pour extraire 
   la sauvegarde du catalogue. D'abord vous devez la localiser en 
   consultant le rapport de la derni\`ere sauvegarde du catalogue. Il comporte 
   des informations qui vous seront pr\'ecieuses pour le restaurer rapidement.
   Par exemple, dans le rapport ci-dessous, les \'elements critiques sont 
   {\it Volume name(s)}, {\it the Volume Session Id} et {\it Volume Session Time}. 
   Si vous les connaissez, vous pouvez restaurer ais\'ement votre catalogue.
\footnotesize
\begin{verbatim}

22-Apr 10:22 HeadMan: Start Backup JobId 7510,
Job=CatalogBackup.2005-04-22_01.10.0
22-Apr 10:23 HeadMan: Bacula 1.37.14 (21Apr05): 22-Apr-2005 10:23:06
  JobId:                  7510
  Job:                    CatalogBackup.2005-04-22_01.10.00
  Backup Level:           Full
  Client:                 Polymatou
  FileSet:                "CatalogFile" 2003-04-10 01:24:01
  Pool:                   "Default"
  Storage:                "DLTDrive"
  Start time:             22-Apr-2005 10:21:00
  End time:               22-Apr-2005 10:23:06
  FD Files Written:       1
  SD Files Written:       1
  FD Bytes Written:       210,739,395
  SD Bytes Written:       210,739,521
  Rate:                   1672.5 KB/s
  Software Compression:   None
  Volume name(s):         DLT-22Apr05
  Volume Session Id:      11
  Volume Session Time:    1114075126
  Last Volume Bytes:      1,428,240,465
  Non-fatal FD errors:    0
  SD Errors:              0
  FD termination status:  OK
  SD termination status:  OK
  Termination:            Backup OK

\end{verbatim}
\normalsize
  Avec ces informations, vous pouvez cr\'eer manuellement un fichier 
  bootstrap et embrayer sur les instructions donn\'ees plus haut pour 
  restaurer le catalogue. Un fichier bootstrap reconstruit pour le job 
  ci-dessus ressemblerait \`a ceci :

\footnotesize
\begin{verbatim}
Volume="DLT-22Apr05"
VolSessionId=11
VolSessionTime=1114075126
FileIndex=1-1
\end{verbatim}
\normalsize    
  
  o\`u les valeurs {\it Volume name}, {\it Volume Session Id} , et {\it Volume Session Time} 
  conformes \`a celles du rapport de job ont \'et\'e introduites. Notez aussi 
  le {\it FileIndex} de valeur 1, ce sera toujours le cas pourvu qu'un seul fichier 
  ait \'et\'e sauvegard\'e par le job.

  L'inconv\'enient de ce fichier bootstrap par rapport \`a celui cr\'e\'e automatiquement 
  lorsque vous le sp\'ecifiez est qu'il ne comporte aucune sp\'ecification {\it File} 
  ou {\it Block}, aussi Bacula doit examiner toutes les donn\'ees du volume pour 
  trouver le fichier requis. Un fichier bootstrap compl\`etement renseign\'e ressemblerait 
  \`a ceci :

\footnotesize
\begin{verbatim}
Volume="DLT-22Apr05"
VolSessionId=11
VolSessionTime=1114075126
VolFile=118-118
VolBlock=0-4053
FileIndex=1-1
\end{verbatim}
\normalsize

\item [Probl\`eme]
   J'essaye de restaurer depuis la derni\`ere sauvegarde compl\`ete en 
   s\'electionnant le choix 3 dans le menu de restauration, puis le 
   JobId \`a restaurer. Bacula affiche alors :

\footnotesize
\begin{verbatim}
   1 Job 0 Files
\end{verbatim}
\normalsize
   et ne restaure rien.
\item[Solution]
   Les enregistrements de fichiers ont tr\`es probablement \'et\'e \'elagu\'es du catalogue 
   soit par expiration de leur p\'eriode de r\'etention (File Retention), soit par purge 
   explicite du job. La commande {\it llist jobid=nn} permet d'obtenir toutes les 
   informations importantes sur ce job :
\footnotesize
\begin{verbatim}
llist jobid=120
           JobId: 120
             Job: save.2005-12-05_18.27.33
        Job.Name: save
     PurgedFiles: 0
            Type: B
           Level: F
    Job.ClientId: 1
     Client.Name: Rufus
       JobStatus: T
       SchedTime: 2005-12-05 18:27:32
       StartTime: 2005-12-05 18:27:35
         EndTime: 2005-12-05 18:27:37
        JobTDate: 1133803657
    VolSessionId: 1
  VolSessionTime: 1133803624
        JobFiles: 236
       JobErrors: 0
 JobMissingFiles: 0
      Job.PoolId: 4
       Pool.Name: Full
   Job.FileSetId: 1
 FileSet.FileSet: BackupSet
\end{verbatim}
\normalsize
   Vous pouvez alors d\'eterminer le(s) volume(s) avec : 
\footnotesize
\begin{verbatim}
sql
select VolumeName from JobMedia,Media where JobId=1 and JobMedia.MediaId=Media.MediaId;
\end{verbatim}
\normalsize
   Finalement, vous pouvez cr\'eer un fichier bootstrap iavec ces informations 
   comme d\'ecrit plus haut.

   A partir de la version 1.38.0, lorsque vous entrez un jobId apr\`es avoir 
   fait le choix 3, Bacula v ous demande si vous voulez restaurer tous les 
   fichiers du job, collecte pour vous les informations requises et \'ecrit le 
   fichier bootstrap.

\item [Probl\`eme]
  Vous n'avez ni fichier bootstrap, ni rapport de job pour votre sauvegarde 
  de catalogue, mais vous avez une sauvegarde et vous savez sur quel volume.
  
\item [Solution]
  Utilisez {\bf bls} pour d\'eterminer o\`u se trouve le fichier requis sur la bande. 
  Par exemple :

\footnotesize
\begin{verbatim}
./bls -j -V DLT-22Apr05 /dev/nst0
\end{verbatim}
\normalsize
  pourrait produire ceci :
\footnotesize
\begin{verbatim}
bls: butil.c:258 Using device: "/dev/nst0" for reading.
21-Jul 18:34 bls: Ready to read from volume "DLT-22Apr05" on device "DLTDrive"
(/dev/nst0).
Volume Record: File:blk=0:0 SessId=11 SessTime=1114075126 JobId=0 DataLen=164
...
Begin Job Session Record: File:blk=118:0 SessId=11 SessTime=1114075126
JobId=7510
   Job=CatalogBackup.2005-04-22_01.10.0 Date=22-Apr-2005 10:21:00 Level=F Type=B
End Job Session Record: File:blk=118:4053 SessId=11 SessTime=1114075126
JobId=7510
   Date=22-Apr-2005 10:23:06 Level=F Type=B Files=1 Bytes=210,739,395 Errors=0
Status=T
...
21-Jul 18:34 bls: End of Volume at file 201 on device "DLTDrive" (/dev/nst0),
Volume "DLT-22Apr05"
21-Jul 18:34 bls: End of all volumes.
\end{verbatim}
\normalsize
  Bien sur, il y aura de nombreuses autres informations affich\'ees, nous n'avons 
  reproduit ici que les essentielles. D'apr\`es les informations sur le d\'ebut 
  ({\it Begin job Session Record} et sur la fin {\it End Job Session Record} du 
  job, vous pouvez \'ecrire un fichier bootstrap comme indiqu\'e plus haut.

\item[Probl\`eme]
  Comment puis-je d\'eterminer o\`u un fichier est stock\'e ?
\item[Solution]
  En principe, ce n'est pas n\'ecessaire. La commande {\bf restore} permet de 
  restaurer la version de la sauvegarde la plus r\'ecente (option 5 du menu), 
  ou une version sauvegard\'ee avant une date donn\'ee (option 8 du menu). Si vous 
  connaissez le JobId du job qui l'a sauvegard\'e, vous pouvez utiliser l'option 3 
  pour entrer ce JobId.

  Pour rechercher le JobId d'une sauvegarde d'un fichier donn\'e, choisissez 
  l'option 2.

  Vous pouvez aussi utiliser la commande {\bf query} pour trouver l'information :

\footnotesize
\begin{verbatim}
*query
Available queries:
     1: List Job totals:
     2: List up to 20 places where a File is saved regardless of the directory:
     3: List where the most recent copies of a file are saved:
     4: List last 20 Full Backups for a Client:
     5: List all backups for a Client after a specified time
     6: List all backups for a Client
     7: List Volume Attributes for a selected Volume:
     8: List Volumes used by selected JobId:
     9: List Volumes to Restore All Files:
    10: List Pool Attributes for a selected Pool:
    11: List total files/bytes by Job:
    12: List total files/bytes by Volume:
    13: List Files for a selected JobId:
    14: List Jobs stored in a selected MediaId:
    15: List Jobs stored for a given Volume name:
Choose a query (1-15):
\end{verbatim}
\normalsize


\end{description}
