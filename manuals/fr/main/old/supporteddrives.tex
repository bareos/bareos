%%
%%

\chapter{Lecteurs de bandes support\'es}
\label{_ChapterStart19}
\index[general]{Lecteurs de bandes support\'es }
\index[general]{Lecteurs!bandes support\'ees }
\addcontentsline{toc}{section}{Lecteurs de bandes support\'es}

M\^eme si votre lecteur est dans la liste ci dessous, v\'erifiez le 
\ilink{Chapitre Test des Lecteurs Bandes}{btape1} de ce manuel
pour les proc\'edures que vous pouvez utiliser pour v\'erifier si votre
lecteur de bandes est susceptible de fonctionner avec Bacula. 
Si votre lecteur est en mode bloc
fixe, il peut para\^itre fonctionner avec Bacula jusqu'\`a ce que vous essayiez de
restaurer et que Bacula tente de se positionner sur la bande. Seuls la
proc\'edure ci-dessus et vos propres tests peuvent vous garantir un
fonctionnement correct. 

Il est tr\`es difficile de fournir une liste de lecteurs de bandes
support\'es, ou de lecteurs qui sont connus pour fonctionner avec Bacula en
raison du peu de retours de la part des usagers. (par cons\'equent, si vous
utilisez Bacula sur un lecteur qui ne figure pas dans la liste, merci de nous
le faire savoir). Selon les informations provenant de nos utilisateurs, les
lecteurs suivants sont connus pour fonctionner avec Bacula. Un trait d'union
dans une colonne signifie "inconnu" : 

\begin{longtable}{|l|l|l|l|l|}
 \hline 
\multicolumn{1}{|c| }{\bf OS  } & \multicolumn{1}{c| }{\bf Fabr.  } &
\multicolumn{1}{c| }{\bf Media  } & \multicolumn{1}{c| }{\bf Mod\`ele  } &
\multicolumn{1}{c| }{\bf Capacit\'e  } \\
 \hline 
{-  } & {ADIC  } & {DLT  } & {Adic Scalar 100 DLT  } & {100GB  } \\
 \hline 
{-  } & {ADIC  } & {DLT  } & {Adic Fastor 22 DLT  } & {-  } \\
 \hline 
{-  } & {-  } & {DDS  } & {Compaq DDS 2,3,4 } & {-  } \\
 \hline 
{-  } & {Exabyte  } & {-  } & {Lecteurs Exabyte de moins de dix ans } & {- 
} \\
 \hline 
{-  } & {Exabyte  } & {-  } & {Exabyte VXA drives  } & {-  } \\
 \hline 
{-  } & {HP  } & {Travan 4  } & {Colorado T4000S  } & {-  } \\
 \hline 
{-  } & {HP  } & {DLT  } & {HP DLT drives  } & {-   } \\
 \hline 
{-  } & {HP  } & {LTO  } & {HP LTO Ultrium drives  } & {-  } \\
 \hline 
{FreeBSD 4.10 RELEASE  } & {HP  } & {DAT  } & {HP StorageWorks DAT72i  } & {- 
} \\
 \hline 
{-  } & {Overland  } & {LTO  } & {LoaderXpress LTO  } & {-  } \\
 \hline 
{-  } & {Overland  } & {-  } & {Neo2000  } & {-  } \\
 \hline 
{-  } & {OnStream  } & {-  } & {OnStream drives (see below)  } & {-  } \\
 \hline 
{-  } & {Quantum  } & {DLT  } & {DLT-8000  } & {40/80GB  } \\
 \hline 
{Linux  } & {Seagate  } & {DDS-4  } & {Scorpio 40  } & {20/40GB  } \\
 \hline 
{FreeBSD 4.9 STABLE  } & {Seagate  } & {DDS-4  } & {STA2401LW  } & {20/40GB  }
\\
 \hline 
{FreeBSD 5.2.1 pthreads patched RELEASE  } & {Seagate  } & {AIT-1  } &
{STA1701W  } & {35/70GB  } \\
 \hline 
{Linux  } & {Sony  } & {DDS-2,3,4  } & {-  } & {4-40GB  } \\
 \hline 
{Linux  } & {Tandberg  } & {-  } & {Tandbert MLR3  } & {-  } \\
 \hline 
{FreeBSD  } & {Tandberg  } & {-  } & {Tandberg SLR6  } & {-  } \\
 \hline 
{Solaris  } & {Tandberg  } & {-  } & {Tandberg SLR75  } & {-  } \\
 \hline 
{Linux Gentoo  } & {ADIC  } & {-  } & {IBM Ultrium LTO I  } & {100/200 Go }
\\ \hline 

\end{longtable}

Une liste des 
\ilink{Librarires support\'ees}{Models} figure dans le 
\ilink{chapitre librairies (autochangers)}{_ChapterStart} de
ce document, o\`u vous trouverez d'autres lecteurs de bandes qui fonctionnent avec
Bacula. 

\section{Lecteurs de bande non support\'es}
\label{UnSupportedDrives}
\index[general]{Lecteurs de bande non support\'es }
\addcontentsline{toc}{section}{Lecteurs de bande non support\'es}

Auparavant les lecteurs de bandes OnStream IDE-SCSI ne fonctionnaient pas avec
Bacula. A partir de la version 1.33 de Bacula et de la version 0.9.14 du
pilote noyau ou sup\'erieur,ce lecteur est support\'e. Consultez le chapitre
de test car vous devez le configurer pour fonctionner en mode blocs de taille
fixe. 

Les lecteurs QIC sont connus pour avoir nombre de particularit\'es (taille de
blocs fixe, et un EOF plut\^ot que deux pour terminer la bande). En
cons\'equence, vous devrez \^etre particuli\`erement attentif \`a sa
configuration pour le faire fonctionner avec Bacula. 

\section{A l'attention des utilisateurs de FreeBSD !!!}
\index[general]{L'attention des utilisateurs de FreeBSD  }
\index[general]{FreeBSD!l'attention des utilisateurs de }
\addcontentsline{toc}{section}{l'attention des utilisateurs de FreeBSD !!!}

A moins que vous n'ayez appliqu\'e un correctif sur la biblioth\`eque pthreads
de votre syst\`eme FreeBSD, vous perdrez des donn\'ees quand Bacula aura
rempli une bande et passera \`a la suivante. La raison en est que les
biblioth\`eques pthreads sans correctifs \'echouent \`a retourner un \'etat
d'alerte \`a Bacula signalant que la fin de bande est proche. Consultez le 
\ilink{chapitre test des lecteurs de bandes}{FreeBSDTapes} de
ce manuel pour d'importantes informations concernant la configuration de votre
lecteur de bande pour qu'il soit compatible avec {\bf Bacula}. 

\section{Librairiess support\'ees}
\index[general]{Librairiess support\'ees }
\addcontentsline{toc}{section}{Librairies support\'ees}

Pour des informations sur les libraries (autochangeurs) support\'ees, allez
voir la section 
\ilink{Libraries connues pour fonctionner avec Bacula}{Models} du chapitre 
Librairies support\'ees de ce manuel. 

\section{Sp\'ecifications des cartouches}
\index[general]{Sp\'ecifications!Cartouches}
\index[general]{Cartouches Sp\'ecifications}
\addcontentsline{toc}{section}{Sp\'ecifications des cartouches}
Nous ne pouvons vraiment pas vous dire quel lecteur acheter pour utiliser Bacula. 
Cependant, nous pouvons recommander d'\'eviter les lecteurs DDS. La 
technologie est plut\^ot ancienne et ces lecteurs n\'ecessitent de fr\'equents 
nettoyages. Les lecteurs DLT sont g\'en\'eralement bien meilleurs (technologie 
plus r\'ecente) et ne requi\`erent pas autant d'op\'erations de nettoyage.

Ci-dessous, vous trouverez une table de sp\'ecifications de cartouches DLT et LTO 
qui vous permettra de vous faire une id\'ee de la capacit\'e et de la rapidit\'e des 
lecteurs et cartouches modernes. Les capacit\'es report\'ees ici sont les natives, 
sans compression. Tous les lecteurs modernes pratiquent la compression 
mat\'erielle, et les fabricants affichent souvent une capacit\'e compress\'ee avec un 
ratio de 2:1. Le facteur de compression r\'eel d\'epend principalement des donn\'ees 
sauvegard\'ees, mais je pense qu'un ratio 1,5:1 est beaucoup plus raisonnable 
(autrement dit, multipliez la valeur de la table par 1,5 pour obtenir une 
estimation grossi\`ere de la capacit\'e compress\'ee). Les taux de transfert sont 
arrondis au GB/hr le plus proche. Les valeurs sont fournies par les divers 
fabricants.

Le type de media est la d\'esignation du fabricant, vous n'\^etes pas oblig\'e de 
l'utiliser (mais vous devriez...) dans vos ressources de configuration Bacula.


\begin{tabular}{|c|c|c|c}
Type de media      & Type de lecteur & Capacit\'e des media & Taux de transfert \\ \hline
DDS-1              & DAT        & 2 GB &        ?? GB/hr   \\ \hline
DDS-2              & DAT        & 4 GB &        ?? GB/hr   \\ \hline
DDS-3              & DAT        & 12 GB &       5.4 GB/hr   \\ \hline
Travan 40          & Travan     & 20 GB &       ?? GB/hr    \\ \hline
DDS-4              & DAT        & 20 GB &       11 GB/hr    \\ \hline
VXA-1              & Exabyte    & 33 GB &       11 GB/hr    \\ \hline
DAT-72             & DAT        & 36 GB &       13 GB/hr    \\ \hline
DLT IV             & DLT8000    & 40 GB  &      22 GB/hr    \\ \hline
VXA-2              & Exabyte    & 80 GB &       22 GB/hr    \\ \hline
Half-high Ultrum 1 & LTO 1      & 100 GB &      27 GB/hr    \\ \hline
Ultrium 1          & LTO 1      & 100 GB &      54 GB/hr    \\ \hline
Super DLT 1        & SDLT 220   & 110 GB &      40 GB/hr    \\ \hline
VXA-3              & Exabyte    & 160 GB &      43 GB/hr    \\ \hline
Super DLT I        & SDLT 320   & 160 GB &      58 GB/hr    \\ \hline
Ultrium 2          & LTO 2      & 200 GB &      108 GB/hr   \\ \hline
Super DLT II       & SDLT 600   & 300 GB &      127 GB/hr   \\ \hline
VXA-4              & Exabyte    & 320 GB &      86 GB/hr    \\ \hline
Ultrium 3          & LTO 3      & 400 GB &      216 GB/hr   \\ \hline
\end{tabular}
