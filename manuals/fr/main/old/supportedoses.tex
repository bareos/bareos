%%
%%

\chapter{Syst\`emes d'exploitation support\'es}
\label{SupportedOSes}
\index[general]{Syst\`emes d'exploitation support\'es }

\begin{itemize}
\item[X] Compl\`etement support\'e
\item[$\star$] Fonctionne, mais non support\'e par le projet Bacula.
\end{itemize}


\begin{tabular}[h]{|l|l|c|c|c|}
  \hline
  OS & Version & Client \small{Daemon} & Director \small{Daemon} & Storage \small{Daemon} \\
  \hline
  \hline
  GNU/Linux
  & All & X & X & X \\
  \hline
  FreeBSD & $\geq$ 5.0 & X & X & X
  \\
  \hline
  Solaris & $\geq$ 8 & X & X & X \\
  \hline
  OpenSolaris & ~ & X & X & X \\
  \hline
  \hline
  MS Windows 32bit& Win98/Me & X  & ~ & ~ \\
  \hline
  ~ & WinNT/2K & X  & $\star$ & $\star$ \\
  \hline
  ~ & XP & X  & $\star$ & $\star$ \\
  ~ & 2008/Vista & X  & $\star$ & $\star$ \\
  MS Windows 64bit& 2008/Vista & X  & ~ & ~ \\
  \hline
  \hline
  MacOS X/Darwin & ~ & X & ~ & ~ \\
  \hline
  OpenBSD & ~ & X & $\star$ & ~ \\
  \hline
  NetBSD & ~ & X & $\star$ & ~ \\
  \hline
  Irix & ~ & $\star$ & ~ & ~ \\
  \hline
  True64 & ~ & $\star$ & ~ & ~ \\
  \hline
  AIX & $\geq$ 4.3 & $\star$ & ~ & ~ \\
  \hline
  BSDI & ~ & $\star$ & ~ & ~ \\
  \hline
  HPUX & ~ & $\star$ & ~ & ~ \\
  \hline
\end{tabular}


\section*{Notes importantes}

\begin{itemize}
\item Par GNU/Linux, nous pensons 32/64bit Gentoo, Red Hat, Fedora, Mandriva,
    Debian, OpenSuSE, Ubuntu, Kubuntu, \dots

  \item FreeBSD (pilote de bande support\'e \`a partir de la version 1.30 --
    allez voir les consid\'erations {\bf importantes} dans la section
    \ilink{Configuration des lecteurs de bandes sur FreeBSD}{FreeBSDTapes} du
    chapitre Test des Bandes de ce manuel.)

\item Le Director et le Storage Daemon MS Windows sont disponibles dans 
  l'installeur du Client.

\item  MacOS X/Darwin (voir 
   \elink{ http://fink.sourceforge.net/}{http://fink.sourceforge.net/} pour
   obtenir les paquets)  
\end{itemize}

Si vous avez un syst\`eme Red Hat r\'ecent ex\'ecutant le noyau 2.4.x et si
vous avez le r\'epertoire {\bf /lib/tls} install\'e sur votre syst\`eme (par
d\'efaut normalement), {\bf Bacula ne fonctionnera pas correctement} Ceci est
d\^u \`a la nouvelle biblioth\`eque pthreads qui est d\'efectueuse. Vous devez
supprimer ce r\'epertoire avant d'ex\'ecuter Bacula, ou vous pouvez simplement
le renommer en {\bf /lib/tls-broken} puis red\'emarrer votre machine (une des
rares occasions o\`u; Linux doit \^etre red\'emarr\'e).  Si vous ne souhaitez
pas d\'eplacer/renommer /lib/tls, une autre alternative est de placer la
variable d'environnement ``LD\_ASSUME\_KERNEL=2.4.19'' avant d'ex\'ecuter
Bacula. Pour cette option, vous n'avez pas besoin de red\'emarrer, et tous les
programmes autres que {\bf Bacula} continueront d'utiliser {\bf /lib/tls}.

Le probl\`eme n'existe pas ur les noyaux 2.6.

Voir le chapitre de Portage de la Documentation Pour Developpeurs pour les
informations concernant le portage sur d'autres syst\`emes.
