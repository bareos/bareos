%%
%%


\chapter{Requerimientos del sistema} \label{SysReqs} \index{Requerimientos del sistema}
\index{Requerimientos del!Sistema} 
\begin{itemize}
\item \textbf{Bacula} ha sido compilado y corre en sistemas OpenSuse Linux, FreeBSD
y Solaris. 
\item Requiere GNU C++, en la versión 2.95 o superior para compilar. En general, el
paquete GNU C++ es adicional al paquete de GNU C, y se necesita que ambos estén
instalados. En sistemas Red Hat, el compilador de C++ es parte del paquete rpm
\textbf{gcc-c++}. 
\item Hay algunos paquetes de terceros que Bacula puede necesitar. A excepción de
MySQL y PostgreSQL que pueden ser hallados en los releases \textbf{depkgs} y
\textbf{depkgs1}. Sin embargo, la mayoría de los sistemas Linux y FreeBSD los
suministran como paquetes del sistema. 
\item Las versiones mínimas para cada una de las bases de datos soportadas son las
siguientes:

\begin{itemize}
\item MySQL 4.1 
\item PostgreSQL 7.4 
\item SQLite 2.8.16 o SQLite 3 
\end{itemize}
\item Si usted desea generar sus propios binarios Win32, por favor revise el archivo
README.mingw32 en el directorio src/win32. El proyecto Bacula realizo una compilación
cruzada del release de Win32 en Linux. De igual manera, se suministra documentación
para la construcción de la versión para Win32, sin embargo, aunque este proceso
puede resultar complejo, esta es de mucha ayuda en este proceso de desarrollo. 
\item \textbf{Bacula} requiere una buena implementación de pthreads (\url{http://en.wikipedia.org/wiki/POSIX_Threads})
para trabajar. Esto no es el caso para algunos de los sistemas BSD. 
\item El código fuente ha sido escrito pensando en la portabilidad y la mayor parte
es compatible con POSIX. Asi, la instalación en cualquier sistema operativo
compatible con POSIX es muy sencilla. 
\item El programa de consola en GNOME fue desarrollado y probado en GNOME 2.x. La
versión de GNOME 1.4 no está soportada por completo. 
\item El programa de consola de wxWidgets fue desarrollado y probado con la ultima
versión estable ANSI o UNICODE de \elink{wxWidgets}{\url{http://www.wxwidgets.org/}}
(2.6.1). Esta aplicación trabaja muy bien con la versión de Windows y GTK+\_2.x
de wxWidgets, y debería funcionar en otras plataformas soportadas por wxWidgets. 
\item El programa de Tray Monitor fue desarrollado en GTK+-2.x. Necesita la versión
de GNOME menor o igual a 2.2, KDE mayor o igual a 3.1 o cualquier manejador
de ventanas que soporte el sistema de bandeja estándar para \elink{ FreeDesktop}{\url{http://www.freedesktop.org/Standards/systemtray-spec}}. 
\item Si se desea activar la edición de linea de comandos y el historial, se necesita
/usr/include/termcap.h y el paquete termcap o la librería ncurses instalada
(libtermcap-devel o ncurses-devel). 
\item Si se desea utilizar DVDs como medios de respaldo, se necesita descargar el
paquete \elink{dvd+rw-tools
5.21.4.10.8}{\url{http://fy.chalmers.se/~appro/linux/DVD+RW/}},
aplicar el patch, ubicado en el directorio \textbf{patches} del árbol de fuentes
original, para hacer que estas herramientas sean compatibles con Bacula, luego
se debe compilarlas e instalarlas. Hay un patch de dvd+rw-tools con versión
6.1, y se espera que el mismo esté integrado en la ultima versión. Se recomienda
no utilizar el utilitario de dvd+rw-tols suministrado con la distribución, a
menos que este seguro que la misma contenga el patch. Las herramientas de dvd+rw-tools
sin el patch instalado, no trabajarán con Bacula. Para finalizar, los dispositivos
de DVD no se recomiendan para respaldos importantes y críticos, debido a su
baja confiabilidad. 
\end{itemize}
