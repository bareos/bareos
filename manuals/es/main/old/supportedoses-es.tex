%%
%%

\chapter{Sistemas Operativos Soportados} \label{SupportedOSes} \index{Sistemas!Operativos Soportados}
\index{Sistemas Operativos Soportados} 
\begin{itemize}
\item [X] Soportado completamente 
\item [$\star$] se ha reportado que trabaja en muchos sistemas, sin embargo, esto
no está garantizado por el proyecto de bacula. 
\end{itemize}
\begin{tabular}{|l|l|c|c|c|}
\hline 
Sistemas Operativos  & Versión  & Cliente {\small {Demonio} }  & Director {\small {Demonio} }  & Storage {\small {Demonio} }\tabularnewline
\hline
\hline 
GNU/Linux  & All  & X  & X  & X \tabularnewline
\hline 
FreeBSD  & $\geq$ 5.0  & X  & X  & X \tabularnewline
\hline 
Solaris  & $\geq$ 8  & X  & X  & X \tabularnewline
\hline 
OpenSolaris  & ~  & X  & X  & X \tabularnewline
\hline
\hline 
MS Windows 32bit  & Win98/Me  & X  & ~  & ~ \tabularnewline
\hline 
~  & WinNT/2K  & X  & $\star$  & $\star$ \tabularnewline
\hline 
~  & XP  & X  & $\star$  & $\star$ \tabularnewline
~  & 2008/Vista  & X  & $\star$  & $\star$ \tabularnewline
MS Windows 64bit  & 2008/Vista  & X  & ~  & ~ \tabularnewline
\hline
\hline 
MacOS X/Darwin  & ~  & X  & ~  & ~ \tabularnewline
\hline 
OpenBSD  & ~  & X  & $\star$  & ~ \tabularnewline
\hline 
NetBSD  & ~  & X  & $\star$  & ~ \tabularnewline
\hline 
Irix  & ~  & $\star$  & ~  & ~ \tabularnewline
\hline 
True64  & ~  & $\star$  & ~  & ~ \tabularnewline
\hline 
AIX  & $\geq$ 4.3  & $\star$  & ~  & ~ \tabularnewline
\hline 
BSDI  & ~  & $\star$  & ~  & ~ \tabularnewline
\hline 
HPUX  & ~  & $\star$  & ~  & ~ \tabularnewline
\hline
\end{tabular}


\section*{Notas importantes}
\begin{itemize}
\item Cuando se hace referencia a GNU/Linux, significa las versiones de 32/64 bits
para Gentoo, Red Hat, Fedora, Mandriva, Debian, OpenSUSE, Ubuntu, Kubuntu, {\ldots{}} 
\item Para versiones menores a la 5.0 de FreeBSD, se deben tomar algunas consideraciones
\textbf{importantes} en los modos para la unidad de tape, descritos en la sección
\ilink{Modos de Tape en FreeBSD} {FreeBSDTapes} del capítulo de Probando
el Tape de este manual. 
\item Los paquetes de director y storage para MS Windows están disponibles como un
programa binario de instalación. 
\item Para MacOSX revisar \elink{http://fink.sourceforge.net/ para obtener los paquetes}{http://fink.sourceforge.net/} 
\end{itemize}
Revise el capítulo de Portabilidad en la guia de Desarrolladores para Bacula,
para obtener información acerca de la portabilidad a otros sistemas.

Si se cuenta con una versión de sistema operativo de Red Hat con kernel de 2.4.x
y se tiene el directorio \textbf{/lib/tls} instalado en el equipo (por defecto),
bacula \textbf{no} correrá. Esto es nuevo para la librería de pthreads y constituye
un defecto. Para correrlo, se debe remover el directorio antes de la ejecución
de bacula, o simplemente, se puede cambiar el nombre a \textbf{/lib/tls-broken},
y luego reiniciar el sistema (una de las pocas veces que Linux debe reiniciarse).
Si no se puede eliminar o renombrar el directorio /lib/tls, un método alternativo
consiste en definir la variable de ambiente \textbf{LD\_ASSUME\_KERNEL=2.4.19},
previo a la ejecución de bacula. Con esta opción, no es necesario el reinicio
y todos los programas, distintos a bacula, continuarán utilizando \textbf{/lib/tls}.
Este problema no ocurre con versiones de kernel 2.6 de Linux. 

