
\chapter{Etiquetas de cintas ANSI e IBM}
\label{AnsiLabelsChapter}
\index[general]{ANSI and IBM Tape Labels} 
\index[general]{Labels!Tape}

Bacula soporta etiquetas de cintas ANSI e IBM si se requieren.
De hecho, con la configuración apropiada, se puede forzar
a Bacula para que exija etiquetas IBM o ANSI.

Bacula puede crear una etiqueta ANSI o IBM, pero si Check Labels
está habilitado (véase mas adelante), Bacula buscará una etiqueta existente, y
si la consigue, mantendrá dicha etiqueta. Por consiguiente, se pueden 
etiquetar las cintas con otros programas, y Bacula los reconocerá 
y soportará.

Incluso, aunque Bacula reconocerá y escribirá etiquetas ANSI e IBM, 
también escribe sus propias etiquetas a los tapes.

Cuando se utiliza el etiquetado ANSI o IBM, se deben restringir
los nombres de volúmenes a un máximo de seis caracteres.

Si ud etiquetó sus volúmenes fuera de Bacula, entonces la 
etiqueta ANSI/IBM únicamente será reconocida si se creó la etiqueta
HDR1 con {\bf BACULA.DATA} en el campo Filename (iniciando con 
el carácter 5). Si Bacula escribe las etiquetas, utilizará 
esta información para reconocer la cinta como un tape de Bacula.
Esto permite a las cintas etiquetadas como ANSI/IBM, ser utilizadas en
sitios con múltiples máquinas y diferentes programas de respaldo.

\section{Directiva Director Pool}

\begin{description}
\item [ Label Type = ANSI | IBM | Bacula]  
  Esta diretiva es implementada en el recurso Director Pool y SD Device.
  Si es especificada en el recurso SD Device, esta tendrá precedencia 
  sobre el valor pasado del Director al SD. El valor por defecto es 
  Label Type = Bacula.
\end{description}

\section{Directivas del dispositivo de Storage Daemon}

\begin{description}
\item [ Label Type = ANSI | IBM | Bacula]  
  Esta diretiva es implementada en el recurso Director Pool y SD Device.
  Si es especificada en el recurso SD Device, esta tendrá precedencia 
  sobre el valor pasado del Director al SD.
 
\item [Check Labels = yes | no]
  Esta directiva se implementa en el recurso SD Device. Si se intenta
  leer etiquetas ANSI o IBM, esta *debe* estar establecida. Incluso si el volumen
  no está etiquetado como ANSI, se puede cambiar a yes, y Bacula chequeará el
  tipo de etiqueta. Sin esta directiva definida como yes, Bacula asumirá que 
  las etiquetas son de tipo Bacula y no verificará si son ANSI o IBM. 
  En otras palabras, si hay alguna posibilidad que Bacula se encuentre con
  una etiqueta ANSI/IBM, este valor debe estar en yes.
\end{description}