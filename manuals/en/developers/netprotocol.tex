%%
%%

\chapter{TCP/IP Network Protocol}
\label{_ChapterStart5}
\index{TCP/IP Network Protocol}
\index{Protocol!TCP/IP Network}
\addcontentsline{toc}{section}{TCP/IP Network Protocol}

\section{General}
\index{General}
\addcontentsline{toc}{subsection}{General}

This document describes the TCP/IP protocol used by Bacula to communicate
between the various daemons and services. The definitive definition of the
protocol can be found in src/lib/bsock.h, src/lib/bnet.c and
src/lib/bnet\_server.c.

Bacula's network protocol is basically a ``packet oriented'' protocol built on
a standard TCP/IP streams. At the lowest level all packet transfers are done
with read() and write() requests on system sockets. Pipes are not used as they
are considered unreliable for large serial data transfers between various
hosts.

Using the routines described below (bnet\_open, bnet\_write, bnet\_recv, and
bnet\_close) guarantees that the number of bytes you write into the socket
will be received as a single record on the other end regardless of how many
low level write() and read() calls are needed. All data transferred are
considered to be binary data.

\section{bnet and Threads}
\index{Threads!bnet and}
\index{Bnet and Threads}
\addcontentsline{toc}{subsection}{bnet and Threads}

These bnet routines work fine in a threaded environment. However, they assume
that there is only one reader or writer on the socket at any time. It is
highly recommended that only a single thread access any BSOCK packet. The
exception to this rule is when the socket is first opened and it is waiting
for a job to start. The wait in the Storage daemon is done in one thread and
then passed to another thread for subsequent handling.

If you envision having two threads using the same BSOCK, think twice, then you
must implement some locking mechanism. However, it probably would not be
appropriate to put locks inside the bnet subroutines for efficiency reasons.

\section{bnet\_open}
\index{Bnet\_open}
\addcontentsline{toc}{subsection}{bnet\_open}

To establish a connection to a server, use the subroutine:

BSOCK *bnet\_open(void *jcr, char *host, char *service, int port,  int *fatal)
bnet\_open(), if successful, returns the Bacula sock descriptor pointer to be
used in subsequent bnet\_send() and bnet\_read() requests. If not successful,
bnet\_open() returns a NULL. If fatal is set on return, it means that a fatal
error occurred and that you should not repeatedly call bnet\_open(). Any error
message will generally be sent to the JCR.

\section{bnet\_send}
\index{Bnet\_send}
\addcontentsline{toc}{subsection}{bnet\_send}

To send a packet, one uses the subroutine:

int bnet\_send(BSOCK *sock) This routine is equivalent to a write() except
that it handles the low level details. The data to be sent is expected to be
in sock-{\textgreater}msg and be sock-{\textgreater}msglen bytes. To send a packet, bnet\_send()
first writes four bytes in network byte order than indicate the size of the
following data packet. It returns:

\footnotesize
\begin{verbatim}
 Returns 0 on failure
 Returns 1 on success
\end{verbatim}
\normalsize

In the case of a failure, an error message will be sent to the JCR contained
within the bsock packet.

\section{bnet\_fsend}
\index{Bnet\_fsend}
\addcontentsline{toc}{subsection}{bnet\_fsend}

This form uses:

int bnet\_fsend(BSOCK *sock, char *format, ...) and it allows you to send a
formatted messages somewhat like fprintf(). The return status is the same as
bnet\_send.

\section{Additional Error information}
\index{Information!Additional Error}
\index{Additional Error information}
\addcontentsline{toc}{subsection}{Additional Error information}

Fro additional error information, you can call {\bf is\_bnet\_error(BSOCK
*bsock)} which will return 0 if there is no error or non-zero if there is an
error on the last transmission. The {\bf is\_bnet\_stop(BSOCK *bsock)}
function will return 0 if there no errors and you can continue sending. It
will return non-zero if there are errors or the line is closed (no more
transmissions should be sent).

\section{bnet\_recv}
\index{Bnet\_recv}
\addcontentsline{toc}{subsection}{bnet\_recv}

To read a packet, one uses the subroutine:

int bnet\_recv(BSOCK *sock) This routine is similar to a read() except that it
handles the low level details. bnet\_read() first reads packet length that
follows as four bytes in network byte order. The data is read into
sock-{\textgreater}msg and is sock-{\textgreater}msglen bytes. If the sock-{\textgreater}msg is not large
enough, bnet\_recv() realloc() the buffer. It will return an error (-2) if
maxbytes is less than the record size sent. It returns:

\footnotesize
\begin{verbatim}
 * Returns number of bytes read
 * Returns 0 on end of file
 * Returns -1 on hard end of file (i.e. network connection close)
 * Returns -2 on error
\end{verbatim}
\normalsize

It should be noted that bnet\_recv() is a blocking read.

\section{bnet\_sig}
\index{Bnet\_sig}
\addcontentsline{toc}{subsection}{bnet\_sig}

To send a ``signal'' from one daemon to another, one uses the subroutine:

int bnet\_sig(BSOCK *sock, SIGNAL) where SIGNAL is one of the following:

\begin{enumerate}
\item BNET\_EOF - deprecated use BNET\_EOD
\item BNET\_EOD - End of data stream, new data may follow
\item BNET\_EOD\_POLL - End of data and poll all in one
\item BNET\_STATUS - Request full status
\item BNET\_TERMINATE - Conversation terminated, doing close()
\item BNET\_POLL - Poll request, I'm hanging on a read
\item BNET\_HEARTBEAT - Heartbeat Response requested
\item BNET\_HB\_RESPONSE - Only response permitted to HB
\item BNET\_PROMPT - Prompt for UA
   \end{enumerate}

\section{bnet\_strerror}
\index{Bnet\_strerror}
\addcontentsline{toc}{subsection}{bnet\_strerror}

Returns a formated string corresponding to the last error that occurred.

\section{bnet\_close}
\index{Bnet\_close}
\addcontentsline{toc}{subsection}{bnet\_close}

The connection with the server remains open until closed by the subroutine:

void bnet\_close(BSOCK *sock)

\section{Becoming a Server}
\index{Server!Becoming a}
\index{Becoming a Server}
\addcontentsline{toc}{subsection}{Becoming a Server}

The bnet\_open() and bnet\_close() routines described above are used on the
client side to establish a connection and terminate a connection with the
server. To become a server (i.e. wait for a connection from a client), use the
routine {\bf bnet\_thread\_server}. The calling sequence is a bit complicated,
please refer to the code in bnet\_server.c and the code at the beginning of
each daemon as examples of how to call it.

\section{Higher Level Conventions}
\index{Conventions!Higher Level}
\index{Higher Level Conventions}
\addcontentsline{toc}{subsection}{Higher Level Conventions}

Within Bacula, we have established the convention that any time a single
record is passed, it is sent with bnet\_send() and read with bnet\_recv().
Thus the normal exchange between the server (S) and the client (C) are:

\footnotesize
\begin{verbatim}
S: wait for connection            C: attempt connection
S: accept connection              C: bnet_send() send request
S: bnet_recv() wait for request
S: act on request
S: bnet_send() send ack            C: bnet_recv() wait for ack
\end{verbatim}
\normalsize

Thus a single command is sent, acted upon by the server, and then
acknowledged.

In certain cases, such as the transfer of the data for a file, all the
information or data cannot be sent in a single packet. In this case, the
convention is that the client will send a command to the server, who knows
that more than one packet will be returned. In this case, the server will
enter a loop:

\footnotesize
\begin{verbatim}
while ((n=bnet_recv(bsock)) > 0) {
   act on request
}
if (n < 0)
   error
\end{verbatim}
\normalsize

The client will perform the following:

\footnotesize
\begin{verbatim}
bnet_send(bsock);
bnet_send(bsock);
...
bnet_sig(bsock, BNET_EOD);
\end{verbatim}
\normalsize

Thus the client will send multiple packets and signal to the server when all
the packets have been sent by sending a zero length record.
