\subsection{VMware Plugin}
\label{VMwarePlugin}
\index[general]{Plugin!VMware}
\index[general]{VMware Plugin}

The \vmware\ Plugin can be
used for agentless backups of virtual machines running on
\vSphere.
It makes use of CBT (Changed Block Tracking) to do space efficient
full and incremental backups, see below for mandatory requirements.

It is included in Bareos since \sinceVersion{fd}{VMware Plugin}{15.2.0}.

\subsubsection{Status}

The Plugin can do full, differential and incremental backup and restore of VM disks.

Current limitations amongst others are:

\limitation*{VMware Plugin}{Normal VM disks can not be excluded from the backup}{%
    It is not yet possible to exclude normal (dependent) VM disks from backups.
    However, independent disks are excluded implicitly because they are not affected
    by snapshots which are required for CBT based backup.}

\limitation*{VMware Plugin}{VM configuration is not backed up}{%
    The VM configuration is not yet (bareos-15.2.2) backed up, so that it is not yet possible to recreate a completely deleted VM.}

\limitation*{VMware Plugin}{Restore can only be done to the same VM}{%
    The restore is only possible to the same existing VM with existing virtual disks.}

\subsubsection{Requirements}
As the Plugin is based on the \vSphere\ Storage APIs for Data Protection,
which requires at least a \vSphere\ Essentials License.
It does not work with standalone unlicensed
VMware® ESXi\trademark.

\subsubsection{Installation}

Install the package \package{bareos-vmware-plugin} including its requirments
by using an appropriate package management tool
(eg. \command{yum}, \command{zypper}, \command{apt})

The \elink{FAQ}{http://www.bareos.org/en/faq.html} may have additional
useful information.

\subsubsection{Configuration}

First add a user account in vCenter that has full privileges by assigning
the account to an administrator role or by adding the account to a group
that is assigned to an administrator role. While any user account
with full privileges could be used, it is better practice to create a separate
user account, so that the actions by this account logged in vSphere are clearly
distinguishable. In the future a more detailed set of required role privilges
may be defined.

When using the vCenter appliance with embedded SSO, a user account usually has the
structure \command{<username>@vsphere.local}, it may be different when using
Active Directory as SSO in vCenter. For the examples here, we will use
\command{bakadm@vsphere.local} with the password \command{Bak.Adm-1234}.

For more details regarding users and permissions in vSphere see
\begin{itemize}
    \item \url{http://pubs.vmware.com/vsphere-55/topic/com.vmware.vsphere.security.doc/GUID-72BFF98C-C530-4C50-BF31-B5779D2A4BBB.html} and
    \item \url{http://pubs.vmware.com/vsphere-55/topic/com.vmware.vsphere.security.doc/GUID-5372F580-5C23-4E9C-8A4E-EF1B4DD9033E.html}
\end{itemize}

Make sure to add or enable the following settings in \file{/etc/bareos/bareos-fd.conf}:

\begin{bconfig}{bareos-fd.conf: Plugin Settings}
...
FileDaemon {
...
  Plugin Directory = /usr/lib/bareos/plugins
  Plugin Names = python
...
}
\end{bconfig}

Note: Depending on Platform, the Plugin Directory may also be \path|/usr/lib64/bareos/plugins|

To define the backup of a VM in Bareos, a job definition and a fileset
resource must be added to the Bareos director confguration.
In vCenter, VMs are usually organized in datacenters and folders.
The following example shows how to configure the backup of the VM
named \textit{websrv1} in the datacenter \textit{mydc1}
folder \textit{webservers} on the vCenter server \command{vcenter.example.org}:

\begin{bconfig}{bareos-dir.conf: VMware Plugin Job and FileSet definition}
Job {
  Name = "vm-websrv1"
  JobDefs = "DefaultJob"
  FileSet = "vm-websrv1_fileset"
}

FileSet {
  Name = "vm-websrv1_fileset"

  Include {
    Options {
         signature = MD5
         Compression = GZIP
    }
    Plugin = "python:module_path=/usr/lib64/bareos/plugins/vmware_plugin:module_name=bareos-fd-vmware:dc=mydc1:folder=/webservers:vmname=websrv1:vcserver=vcenter.example.org:vcuser=bakadm@vsphere.local:vcpass=Bak.Adm-1234"
  }
}
\end{bconfig}

For VMs defined in the root-folder, \command{folder=/} must be specified
in the Plugin definition.

\subsubsection{Backup}

Before running the first backup, CBT (Changed Block Tracking) must be
enabled for the VMs to be backed up.

As of \url{http://kb.vmware.com/kb/2075984} manually enabling CBT is
currently not working properly. The API however works properly.
To enable CBT use the Script \command{vmware_cbt_tool.py}, it is packaged
in the bareos-vmware-plugin package:

\begin{commands}{usage of vmware\_cbt\_tool.py}
# <parameter>vmware_cbt_tool.py --help</parameter>
usage: vmware_cbt_tool.py [-h] -s HOST [-o PORT] -u USER [-p PASSWORD] -d
                          DATACENTER -f FOLDER -v VMNAME [--enablecbt]
                          [--disablecbt] [--resetcbt] [--info]

Process args for enabling/disabling/resetting CBT

optional arguments:
  -h, --help            show this help message and exit
  -s HOST, --host HOST  Remote host to connect to
  -o PORT, --port PORT  Port to connect on
  -u USER, --user USER  User name to use when connecting to host
  -p PASSWORD, --password PASSWORD
                        Password to use when connecting to host
  -d DATACENTER, --datacenter DATACENTER
                        DataCenter Name
  -f FOLDER, --folder FOLDER
                        Folder Name
  -v VMNAME, --vmname VMNAME
                        Names of the Virtual Machines
  --enablecbt           Enable CBT
  --disablecbt          Disable CBT
  --resetcbt            Reset CBT (disable, then enable)
  --info                Show information (CBT supported and enabled or
                        disabled)
\end{commands}

For the above configuration example, the command to enable CBT would be

\begin{commands}{Example using vmware\_cbt\_tool.py}
# <parameter>vmware_cbt_tool.py -s vcenter.example.org -u bakadm@vsphere.local -p Bak.Adm-1234 -d mydc1 -f /webservers -v websrv1 --enablecbt</parameter>
\end{commands}

Note: CBT does not work if the virtual hardware version is 6 or earlier.

After enabling CBT, Backup Jobs can be run or scheduled as usual,
for example in \command{bconsole}:

\bcommand{run}{job=vm-websrv1 level=Full}

\subsubsection{Restore}

For restore, the VM must be powered off and no snapshot must exist.
In \command{bconsole} use the restore menu 5, select the correct FileSet
and enter \bcommand{mark}{*}, then \bcommand{}{done}. After restore has finished,
the VM can be powered on.
