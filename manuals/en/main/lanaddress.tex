   \section{Using different IP Adresses for SD -- FD Communication}
   \label{LanAddress}
   \index[general]{Lan Address}

The \configdirective{Lan Address} configuration directive is needed to support a network
topology where client and storage are situated inside of a LAN,  but the Director is
outside of that LAN in the internet and accesses SD and FD via SNAT / port forwarding.

Consider the following scheme:

\begin{commands}{}
   /-------------------\
   |                   |    LAN 10.0.0.1/24
   |                   |
   |  FD_LAN   SD_LAN  |
   |  .10         .20  |
   |                   |
   \___________________/
             |
         NAT Firewall
         FD: 8.8.8.10 -> 10.0.0.10
         SD: 8.8.8.20 -> 10.0.0.20
             |
   /-------------------\
   |                   |
   |                   |     WAN / Internet
   |        DIR        |
   |     8.8.8.100     |
   |                   |
   | FD_WAN   SD_WAN   |
   | .30         .40   |
   \___________________/
   \end{commands}

The director can access the FD\_LAN via the IP 8.8.8.10, which is
forwarded to the IP 10.0.0.10 inside of the LAN.

The director can access the SD\_LAN via the IP 8.8.8.20 which is
forwarded to the IP 10.0.0.20 inside of the LAN.

There is also a FD and a SD outside of the LAN, which have the IPs
8.8.8.30 and 8.8.8.40

All resources are configured so that the \configdirective{Address} directive gets the
Address where the Director can reach the daemons.

Additionally, devices being in the LAN get the LAN address configured in
the \configdirective{LanAddress} directive:

The configuration looks as follows:

\begin{bconfig}{bareos-dir.conf}
Client {
   Name = FD_LAN
   Address = 8.8.8.10
   LanAddress = 10.0.0.10
   ...
}

Storage {
   Name = SD_LAN
   Address = 8.8.8.20
   LanAddress = 10.0.0.20
   ...
}

Client {
   Name = FD_WAN
   Address = 8.8.8.30
   ...
}

Storage {
   Name = SD_WAN
   Address = 8.8.8.40
   ...
}
\end{bconfig}

This way, backups and restores from each FD using each SD are possible
as long as the firewall allows the needed network connections.

The director simply checks if both the involved client and storage
both have a \configdirective{LanAddress} configured.

In that case, the initiating daemon is ordered to connect to the \configdirective{LanAddress}
instead of the \configdirective{Address}. (In active client mode, the FD connects to the SD, in
passive client mode the SD connects to the FD).

If only one or none of the involved client and storage have a \configdirective{LanAddress}
configured, the \configdirective{Address} is used as connection target for the
initiating daemon.
