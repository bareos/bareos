\label{releasenotes}

The technical changelog is automatically generated from the Bareos bug tracking system, see \url{http://bugs.bareos.org/changelog\_page.php}.

Please note, that some of the subreleases are only internal development releases.

Open issues for a specific version are shown at
\url{http://bugs.bareos.org/roadmap\_page.php}.

The overview about new feature of a release are shown at
\url{https://github.com/bareos/bareos} and in the \ilink{index}{index} of this document.

This chapter concentrates on things to do when updating an existing Baroes installation.

\warning{While all the source code is published on \elink{GitHub}{https://github.com/bareos/bareos}, the releases of packages on \url{http://download.baroes.org} is limited to the initial versions of a major release. Later maintenance releases are only published on \url{https://download.bareos.com}.}

\section*{Bareos-15.2}

\releasenote{15.2.1}{

\begin{tabular}{p{0.2\textwidth} p{0.8\textwidth}}
Code Release      & TODO \\
Database Version  & 2004\\
                  & Database update required, see the \nameref{bareos-update} section.\\
Release Ticket    & \ticket{501} \\
Url               & \releaseUrlDownloadBareosOrg{15.2} \\
\end{tabular}

Beta release.

\begin{itemize}
    \item The default setting for the Bacula Compatbile mode in  \linkResourceDirective{Fd}{Client}{Compatible} and \linkResourceDirective{Sd}{Storage}{Compatible} have been changed from \parameter{yes} to \parameter{no}.
    \item The configuration syntax for Storage Daemon Cloud Backends Ceph and GlusterFS have changed.
	Before bareos-15.2, options have been configured as part of the \linkResourceDirective{Sd}{Device}{Archive Device} directive, while now the Archive Device contains only information text and options are defined via the \linkResourceDirective{Sd}{Device}{Device Options} directive. See examples in \linkResourceDirective{Sd}{Device}{Device Options}.
% # Old syntax:
% #    Archive Device = /etc/ceph/ceph.conf:poolname
% #
% # New syntax:
% #    Archive Device = <text>
% #    Device Options = "conffile=/etc/ceph/ceph.conf,poolname=poolname"
\end{itemize}

}


\section*{Bareos-14.2}

It is known, that \command{drop_database} scripts will not longer work on PostgreSQL $<$ 8.4. However, as \command{drop_database} scripts are very seldom needed, package dependencies do not yet enforce PostgreSQL $>=$ 8.4.
We plan to ensure this in future version of Bareos.

\releasenote{14.2.5}{

\begin{tabular}{p{0.2\textwidth} p{0.8\textwidth}}
Code Release      & 2015-06-01\\
Database Version  & 2003 (unchanged)\\
Release Ticket    & \ticket{447} \\
Url               & \releaseUrlDownloadBareosCom{14.2} \\
\end{tabular}

This release contains several bugfixes and added the platforms \os{Debian}{8} and \os{Fedora}{21}.
}


\releasenote{14.2.4}{

\begin{tabular}{p{0.2\textwidth} p{0.8\textwidth}}
Code Release      & 2015-03-23 \\
Database Version  & 2003 (unchanged)\\
Release Ticket    & \ticket{420} \\
Url               & \releaseUrlDownloadBareosCom{14.2} \\
\end{tabular}

This release contains several bugfixes, including one major bugfix (\ticket{437}), relevant for those of you using backup to disk with autolabeling enabled.

It can lead to loss of a 64k block of data when all of this conditions apply:
\begin{itemize}
 \item backups are written to disk (tape backups are not affected)
 \item autolabelling is enabled
 \item a backup spans over multiple volumes
 \item the additional volumes are newly created and labeled during the backup
\end{itemize}
If existing volumes are used for backups spanning over multiple volumes, the problem does not occur.

We recommend to update to the latest packages as soon as possible.

If an update is not possible immediately,
autolabeling should be disabled and volumes should be labelled manually
until the update can be installed. 

If you are affected by the 64k bug, we recommend that you schedule a full backup after fixing the problem in order to get a
proper full backup of all files.
}

\releasenote{14.2.3}{

\begin{tabular}{p{0.2\textwidth} p{0.8\textwidth}}
Code Release      & 2015-02-02 \\
Database Version  & 2003 (unchanged)\\
Release Ticket    & \ticket{393}\\
Url               & \releaseUrlDownloadBareosCom{14.2} \\
\end{tabular}

}

\releasenote{14.2.2}{

\begin{tabular}{p{0.2\textwidth} p{0.8\textwidth}}
Code Release      & 2014-12-12 \\
Database Version  & 2003 (unchanged)\\
                  & Database update required if updating from version $<$ 14.2.\\
                  & See the \nameref{bareos-update} section for details.\\
Url               & \releaseUrlDownloadBareosOrg{14.2} \\
                  & \releaseUrlDownloadBareosCom{14.2} \\
\end{tabular}

First stable release of the Bareos 14.2 branch.
}

\releasenote{14.2.1}{

\begin{tabular}{p{0.2\textwidth} p{0.8\textwidth}}
Code Release & 2014-09-22 \\
Database Version  & 2003\\
                  & Database update required, see the \nameref{bareos-update} section.\\
Url               & \releaseUrlDownloadBareosOrg{14.2} \\
\end{tabular}

Beta release.
}

\section*{Bareos-13.2}

\releasenote{13.2.4}{

\begin{tabular}{p{0.2\textwidth} p{0.8\textwidth}}
Code Release      & 2014-11-05 \\
Database Version  & 2002 (unchanged)\\
Url               & \releaseUrlDownloadBareosCom{13.2} \\
\end{tabular}
}

\releasenote{13.2.3}{

\begin{tabular}{p{0.2\textwidth} p{0.8\textwidth}}
Code Release      & 2014-03-11 \\
Database Version  & 2002\\
                  & Database update required, see the \nameref{bareos-update} section.\\
Url               & \releaseUrlDownloadBareosCom{13.2} \\
\end{tabular}

It is known, that \command{drop_database} scripts will not longer work on PostgreSQL $<$ 8.4. However, as \command{drop_database} scripts are very seldom needed, package dependencies do not yet enforce PostgreSQL $>=$ 8.4. We plan to ensure this in future version of Bareos.
}

\releasenote{13.2.2}{

\begin{tabular}{p{0.2\textwidth} p{0.8\textwidth}}
Code Release      & 2013-11-19 \\
Database Version  & 2001 (unchanged)\\
Url               & \releaseUrlDownloadBareosOrg{13.2} \\
                  & \releaseUrlDownloadBareosCom{13.2} \\
\end{tabular}
}



\section*{Bareos-12.4}


\releasenote{12.4.6}{

\begin{tabular}{p{0.2\textwidth} p{0.8\textwidth}}
Code Release      & 2013-11-19 \\
Database Version  & 2001 (unchanged)\\
Url               & \releaseUrlDownloadBareosOrg{12.4} \\
                  & \releaseUrlDownloadBareosCom{12.4} \\
\end{tabular}
}



\releasenote{12.4.5}{

\begin{tabular}{p{0.2\textwidth} p{0.8\textwidth}}
Code Release      & 2013-09-10 \\
Database Version  & 2001 (unchanged)\\
Url               & \releaseUrlDownloadBareosCom{12.4} \\
\end{tabular}
}


\releasenote{12.4.4}{

\begin{tabular}{p{0.2\textwidth} p{0.8\textwidth}}
Code Release      & 2013-06-17 \\
Database Version  & 2001 (unchanged)\\
Url               & \releaseUrlDownloadBareosOrg{12.4} \\
                  & \releaseUrlDownloadBareosCom{12.4} \\
\end{tabular}
}


\releasenote{12.4.3}{

\begin{tabular}{p{0.2\textwidth} p{0.8\textwidth}}
Code Release      & 2013-04-15 \\
Database Version  & 2001 (unchanged)\\
Url               & \releaseUrlDownloadBareosOrg{12.4} \\
                  & \releaseUrlDownloadBareosCom{12.4} \\
\end{tabular}
}


\releasenote{12.4.2}{

\begin{tabular}{p{0.2\textwidth} p{0.8\textwidth}}
Code Release      & 2013-03-03 \\
Database Version  & 2001 (unchanged)\\
\end{tabular}
}

\releasenote{12.4.1}{

\begin{tabular}{p{0.2\textwidth} p{0.8\textwidth}}
Code Release      & 2013-02-06 \\
Database Version  & 2001 (initial)\\
\end{tabular}

This have been the initial release of Bareos.

Information about migrating from Bacula to Bareos are available at \elink{Howto upgrade from Bacula to Bareos}{http://www.bareos.org/en/HOWTO/articles/upgrade_bacula_bareos.html} and in section \nameref{compat-bacula}.
}
