\chapter{Backward Compatibility}
\index[general]{Backward Compatibility}
One of the major goals of Backup softare is to ensure that you can restore
tapes (the word tape should also include disk volumes) that you wrote years
ago.  This means that each new version of the software should be able to read old
format tapes. The first problem you will have is to ensure that the
hardware is still working some years down the road, and the second
problem will be to ensure that the media will still be good, then
your OS must be able to interface to the device, and finally Bareos
must be able to recognize old formats.  All the problems except the
last are ones that we cannot solve, but by careful planning you can.

Since the very beginning of Bacula (January 2000) until today (2013),
there have been three major Bacula/Bareos tape formats.  The second format
was introduced in Bacula version 1.27 in November of 2002.
Bareos has been required to adapt the tape format to avoid potential trademark issues,
but is able to read also the old Bacula tape formats.

Though the tape format is basically fixed, the kinds of data that we can put on the
tapes are extensible, and that is how we added new features
such as ACLs, Win32 data, encrypted data, ...  Obviously, an older
version of Bacula/Bareos would not know how to read these newer data streams,
but each newer version of Bareos should know how to read all the
older streams.

\TODO{add information about bareos bacula compatibility}

% If you want to be 100% sure that you can read old tapes, you
% should:
%
% 1. Try reading old tapes from time to time -- e.g. at least once
% a year.
%
% 2. Keep statically linked copies of every version of Bareos that you use
% in production then if for some reason, we botch up old tape compatibility, you
% can always pull out an old copy of Bareos ...
%
% The second point is probably overkill but if you want to be sure, it may
% save you someday.
