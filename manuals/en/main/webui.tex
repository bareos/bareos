%%
%%

This chapter addresses the installation process of the Bareos Webui.

Since \sinceVersion{dir}{bareos-webui}{15.2.0} bareos-webui is part of the Bareos project and available for a number of platforms.

\begin{center}
  \includegraphics[width=0.8\textwidth]{\idir bareos-webui-jobs}
\end{center}

\section{System Requirements}

\begin{itemize}
\item A working Bareos environment. The \bareosDir version should match the \bareosWebui version.
\item A Bareos platform, where bareos-webui packages are provided.
\item You can install the \bareosWebui on any host it does not have to be installed on the same as the \bareosDir.
\item The default installation usese an Apache 2.x Webserver with mod-rewrite, mod-php5 and mod-setenv.
\item PHP $>$= 5.3.3
\end{itemize}

\subsection{Version $<$ 16.2}

\bareosWebui $>$= 16.2.4 includes the required Zend Framework 2 components. For older versions, you must install Zend Framework separatly.

Following is a list where to get Zend Framework 2 packages from, if not already included in the repository of your distribution.

\begin{itemize}
\item Zend Framework 2.2.x or later.
  \textbf{Note:} Unfortunately, not all distributions offer a Zend Framework 2 package.
  The following list shows where to get the Zend Framework 2 package.
  \begin{itemize}
  \item RHEL, CentOS
    \begin{itemize}
    \item \url{https://fedoraproject.org/wiki/EPEL}
    \item \url{https://apps.fedoraproject.org/packages/php-ZendFramework2}
    \end{itemize}

  \item Fedora
    \begin{itemize}
    \item \url{https://apps.fedoraproject.org/packages/php-ZendFramework2}
    \end{itemize}

  \item SUSE, Debian, Ubuntu
    \begin{itemize}
    \item \url{http://download.bareos.org/bareos}
    \end{itemize}
  \end{itemize}
\end{itemize}

\subsection

\section{Installation}

\subsection{Adding the Bareos Repository}

If not already done, add the Bareos repository that is matching your Linux distribution. Please have a look at the chapter \nameref{sec:InstallBareosPackages} for more information on how to achieve this.

\subsection{Install the bareos-webui package}

After adding the repository simply install the bareos-webui package via your package manager.

\begin{itemize}
 \item RHEL, CentOS and Fedora
\begin{commands}{}
yum install bareos-webui
\end{commands}
 or
\begin{commands}{}
dnf install bareos-webui
\end{commands}
\end{itemize}

\begin{itemize}
 \item SUSE Linux Enterprise Server (SLES), openSUSE
\begin{commands}{}
zypper install bareos-webui
\end{commands}
\end{itemize}

\begin{itemize}
 \item Debian, Ubuntu
\begin{commands}{}
apt-get install bareos-webui
\end{commands}
\end{itemize}

\subsection{Configuration of restricted consoles and profile resources}

You can have multiple consoles with different names and passwords, sort of like multiple users, each with different privileges. As a default, these consoles can do absolutely nothing – no commands whatsoever. You give them privileges or rather access to commands and resources by specifying access control lists (ACLs) in the director’s console resource. The ACLs are specified by a directive followed by a list of access names.

It is required to add at least one restricted named console. The restricted named consoles are used for authentication and access control. The name and password directives of the restricted consoles are the credentials you have to provide during authentication to the webui as username and password.

The bareos-webui package provides a default console and profile configuration.

\file{/etc/bareos/bareos-dir.d/console/admin.conf.example}
\file{/etc/bareos/bareos-dir.d/profile/webui-admin.conf}

Please note, that in older releases before 16.2 those configuration files were provided in a different location and had to be included manually.

\begin{commands}{add webui-consoles and webui-profiles to the Bareos Director configuration}
echo "@/etc/bareos/bareos-dir.d/webui-consoles.conf" >> /etc/bareos/bareos-dir.conf
echo "@/etc/bareos/bareos-dir.d/webui-profiles.conf" >> /etc/bareos/bareos-dir.conf
\end{commands}

\file{/etc/bareos/bareos-dir.d/console/admin.conf.example} provides a default admin console example using the default webui-admin profile we provide. Please create a console like that to be able to authenticate via webui against the \bareosDir.

\begin{bconfig}{/etc/bareos/bareos-dir.d/console/admin.conf.example}
Console {
  Name = admin
  Password = "admin"
  Profile = webui-admin
}
\end{bconfig}

For more details about the console resource configuration, please have a look at the chapter \nameref{DirectorResourceConsole}.

\file{/etc/bareos/bareos-dir.d/profile/webui-admin.conf} provides a default admin profile example which allows nearly every command available to use, except a few you really should use with care.

\begin{bconfig}{/etc/bareos/bareos-dir.d/profile/webui-admin.conf}
Profile {
  Name = webui-admin
  CommandACL = !.bvfs_clear_cache, !.exit, !.sql, !configure, !create, !delete, !purge, !sqlquery, !umount, !unmount, *all*
  Job ACL = *all*
  Schedule ACL = *all*
  Catalog ACL = *all*
  Pool ACL = *all*
  Storage ACL = *all*
  Client ACL = *all*
  FileSet ACL = *all*
  Where ACL = *all*
  Plugin Options ACL = *all*
}
\end{bconfig}

For more details about profile resource configuration in bareos, please have a look at the chapter \nameref{DirectorResourceProfile}.

\warning{Do not forget to reload your new director configuration.}

\subsection{Configure your Apache Webserver}

\index[general]{Apache!bareos-webui}

If you have installed from package, a default configuration is provided, please see \file{/etc/apache2/conf.d/bareos-webui.conf}, \file{/etc/httpd/conf.d/bareos-webui.conf} or \file{/etc/apache2/available-conf/bareos-webui.conf}.

The required Apache modules, \argument{setenv}, \argument{rewrite} and \argument{php} are enabled via package postinstall script. You simply need to restart your apache webserver manually.

\subsection{Configure your /etc/bareos-webui/directors.ini}
\index[general]{Configuration!WebUI}

Configure your directors in \file{/etc/bareos-webui/directors.ini} to match your settings, which you have chosen in the previous steps.

The configuration file \file{/etc/bareos-webui/directors.ini} should look similar to this:

\begin{bconfig}{Bareos-webui directors.ini}
;
; Bareos WebUI Configuration
; File: /etc/bareos-webui/directors.ini
;

;
; Section bareos-dir
;
[bareos-dir]

; Enable or disable director. Possible values are "yes" or "no", the default is "yes".
enabled = "yes"

; Fill in the IP-Address or FQDN of your director.
diraddress = "localhost"

; Default value is 9101
dirport = 9101

; Set catalog to explicit value if you have multiple catalogs,
; the default value is "MyCatalog".
;catalog = "MyCatalog"

;
; Section remote-dir
;
[remote-dir]
enabled = "no"
diraddress = "hostname"
;dirport = 9101
;catalog = "MyCatalog"
;tls_verify_peer = false
;server_can_do_tls = false
;server_requires_tls = false
;client_can_do_tls = false
;client_requires_tls = false
;ca_file = ""
;cert_file = ""
;cert_file_passphrase = ""
;allowed_cns = ""
\end{bconfig}

You can add as many directors as you want, also the same host with a different name and different catalog, if you have multiple catalogs.

\subsection{Configure your /etc/bareos-webui/configuration.ini}

Since 16.2 you are able to configure a few parameters of the webui to your needs.

\begin{bconfig}{Bareos-webui configuration.ini}
[session]
# Default: 3600 seconds
timeout=3600

[tables]
# Define a list of pagination values.
# Default: 10,25,50,100
pagination_values=10,25,50,100

# Default number of rows per page
# for possible values see pagination_values
# Default: 25
pagination_default_value=25

# State saving - restore table state on page reload.
# Default: false
save_previous_state=false

[autochanger]
# Pooltype for label to use as filter.
# See pooltype in output of bconsole: list pools
# Default: none
labelpooltype=scratch
\end{bconfig}

You are now able to login by calling http://HOSTNAME/bareos-webui in your browser of choice. Remember, login credentials are defined in your Bareos Director Console configuration.

\section{Additional information}

\subsection{SELinux}
\index[general]{SELinux!bareos-webui}

To install bareos-webui on a system with SELinux enabled, the following additional steps have to be performed.
\begin{itemize}
 \item Allow HTTPD scripts and modules to connect to the network
\begin{commands}{}
setsebool -P httpd_can_network_connect on
\end{commands}
\end{itemize}

\subsection{NGINX}
\index[general]{nginx!bareos-webui}

If you prefer to use bareos-webui on Nginx with php5-fpm instead of Apache,
a basic working configuration could look like this:

\begin{bconfig}{bareos-webui on nginx}
server {

        listen       9100;
        server_name  bareos;
        root         /var/www/bareos-webui/public;

        location / {
                index index.php;
                try_files $uri $uri/ /index.php?$query_string;
        }

        location ~ .php$ {

                include snippets/fastcgi-php.conf;

                # With php5-cgi alone pass the PHP
                # scripts to FastCGI server
                # listening on 127.0.0.1:9000

                # fastcgi_pass 127.0.0.1:9000;

                # With php5-fpm:

                fastcgi_pass unix:/var/run/php5-fpm.sock;

                # Set APPLICATION_ENV to either 'production' or 'development'

                # fastcgi_param APPLICATION_ENV development;
                fastcgi_param APPLICATION_ENV production;

        }

}
\end{bconfig}
