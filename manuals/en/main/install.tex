If you are like me, you want to get Bareos running immediately to get a feel
for it, then later you want to go back and read about all the details. This
chapter attempts to accomplish just that: get you going quickly without all
the details.

Bareos comes prepackaged for a number of Linux distributions.
So the easiest way to get to a running Bareos installation, 
is to use a platform where prepacked Bareos packages are available.
Additional information can be found in the chapter \ilink{Operating Systems}{SupportedOSes}.

%\TODO{If you are using a platform, for which Bareos is not available in a prepackaged format,
%please refer to the \ilink{Building chapter}{compile}.}

If Bareos is available as a package, 
only 5 steps are required to get to a running Bareos System:
\begin{enumerate}
    \item \nameref{sec:AddSoftwareRepository}
    \item \nameref{sec:ChooseDatabaseBackend}
    \item \nameref{sec:InstallBareosPackages}
    \item \nameref{sec:CreateDatabase}
    \item \nameref{sec:StartDaemons}
\end{enumerate}

This will start a very basic Bareos installation which will regularly backup a directory to disk.
In order to fit it to your needs, you'll have to adapt the configuration and might want to backup other clients.

\section{Decide about the Bareos release to use}
    \label{sec:AddSoftwareRepository}

\begin{itemize}
   \item \url{http://download.bareos.org/bareos/release/latest/}
\end{itemize}

You'll find Bareos binary package repositories at \url{http://download.bareos.org/}.
The lastet stable released version is available at \url{http://download.bareos.org/bareos/release/latest/}.

The public key to verify the repository is also in repository directory
(\file{Release.key} for Debian based distributions, \file{repodata/repomd.xml.key} for RPM based distributions).

Section \nameref{sec:InstallBareosPackages} describes how to add the software repository to your system.


\section{Decide about the Database Backend}
    \label{sec:ChooseDatabaseBackend}

Next you have to decide, what database backend you want to use.
Bareos supports following database backends:
\begin{itemize}
    \item PostgreSQL by package \package{bareos-database-postgresql}
    \item MySQL by package \package{bareos-database-mysql}
    \item Sqlite by package \package{bareos-database-sqlite3} \\
        \warning{The Sqlite backend is only intended for testing, not for productive use.}
\end{itemize}

The PostgreSQL backend is the default.
However, the MySQL backend is also supported,
while the Sqlite backend is intended for testing purposes only.

The Bareos database packages have there dependencies only to the database client packages, 
therefore the database itself must be installed manually.


\section{Install the Bareos Software Packages}
    \label{sec:InstallBareosPackages}

You will have to install the package \package{bareos} 
and the database backend package (\package{bareos-database-*}) you want to use.
The corresponding database should already be installed and running, see \nameref{sec:ChooseDatabaseBackend}.

If you do not explicitly choose a database backend, your operating system installer will choose one for you.
The default should be PostgreSQL, but depending on your operating system and the already installed packages, 
this may differ.

The package \package{bareos} is only a meta package, that contains dependencies to the main components of Bareos, see \nameref{sec:BareosPackages}. 
If you want to setup a distributed environment (like one Director, separate database server, multiple Storage daemons)
you have to choose the corresponding Bareos packages to install on each hosts instead of just installing the \package{bareos} package.


\subsection{Install on RedHat based Linux Distributions}

\subsubsection{RHEL$\ge$6, CentOS$\ge$6, Fedora}
\index[general]{Platform!RHEL}
\index[general]{Platform!CentOS}
\index[general]{Platform!Fedora}

\begin{commands}{Bareos installation on RHEL $\ge$ 6 / CentOS $\ge$ 6 / Fedora}
#
# define parameter
#

DIST=RHEL_7
# or
# DIST=RHEL_6
# DIST=Fedora_20
# DIST=CentOS_7
# DIST=CentOS_6

DATABASE=postgresql
# or
# DATABASE=mysql

# add the Bareos repository
URL=http://download.bareos.org/bareos/release/latest/$DIST
wget -O /etc/yum.repos.d/bareos.repo $URL/bareos.repo

# install Bareos packages
yum install bareos bareos-database-$DATABASE
\end{commands}
\hide{$}


\subsubsection{RHEL 5, CentOS 5}
\index[general]{Platform!RHEL!5}
\index[general]{Platform!CentOS!5}

yum in RHEL 5/CentOS 5 has slightly different behaviour as far as dependency resolving is concerned: it sometimes install a dependent package after the one that has the dependency defined. To make sure that it works, install the desired Bareos database backend package first in a separate step:

\begin{commands}{Bareos installation on RHEL 5 / CentOS 5}
#
# define parameter
#

DIST=RHEL_5
# or
# DIST=CentOS_5

DATABASE=postgresql
# or
# DATABASE=mysql

# add the Bareos repository
URL=http://download.bareos.org/bareos/release/latest/$DIST
wget -O /etc/yum.repos.d/bareos.repo $URL/bareos.repo

# install Bareos packages
yum install bareos-database-$DATABASE
yum install bareos
\end{commands}
\hide{$}

\subsection{Install on SUSE based Linux Distributions}

\subsubsection{SUSE Linux Enterprise Server (SLES), openSUSE}
\index[general]{Platform!SLES}
\index[general]{Platform!openSUSE}

\begin{commands}{Bareos installation on SLES / openSUSE}
#
# define parameter
#

DIST=SLE_12
# or
# DIST=SLE_11_SP3
# DIST=openSUSE_13.1

DATABASE=postgresql
# or
# DATABASE=mysql

# add the Bareos repository
URL=http://download.bareos.org/bareos/release/latest/$DIST
zypper addrepo --refresh $URL/bareos.repo

# install Bareos packages
zypper install bareos bareos-database-$DATABASE
\end{commands}
\hide{$}



\subsection{Install on Debian based Linux Distributions}

\subsubsection{Debian / Ubuntu}
\index[general]{Platform!Debian}
\index[general]{Platform!Ubuntu}

\begin{commands}{Bareos installation on Debian / Ubuntu}
#
# define parameter
#

DIST=Debian_8.0
# or
# DIST=Debian_7.0
# DIST=Debian_6.0
# DIST=xUbuntu_14.04
# DIST=xUbuntu_12.04

DATABASE=postgresql
# or
# DATABASE=mysql

URL=http://download.bareos.org/bareos/release/latest/$DIST/

# add the Bareos repository
printf "deb $URL /\n" > /etc/apt/sources.list.d/bareos.list

# add package key
wget -q $URL/Release.key -O- | apt-key add -

# install Bareos packages
apt-get update
apt-get install bareos bareos-database-$DATABASE
\end{commands}

If you prefer using the versions of Bareos directly integrated into the distributions, 
please note that there are some differences, see \nameref{sec:DebianOrgLimitations}.

\section{Prepare Bareos database}
    \label{sec:CreateDatabase}

We assume that you have already your database installed and basically running.
Currently the database backend PostgreSQL and MySQL are recommended. The Sqlite database backend is only intended for testing purposes.

The easiest way to set up a database is using an system account that have passwordless local access to the database. 
Often this is the user \user{root} for MySQL and the user \user{postgres} for PostgreSQL.

For details, see chapter \nameref{CatMaintenanceChapter}.

\subsection{Debian based Linux Distributions}

Since Bareos \sinceVersion{dir}{dbconfig-common (Debian)}{14.2.0} the Debian (and Ubuntu) based packages support the \package{dbconfig-common} mechanism to create and update the Bareos database.

Follow the instructions during install to configure it according to your needs.

\begin{center}
\includegraphics[width=0.45\textwidth]{\idir dbconfig-1-enable}
\includegraphics[width=0.45\textwidth]{\idir dbconfig-2-select-database-type}
\end{center}

If you decide not to use \package{dbconfig-common} (selecting \parameter{<No>} on the initial dialog), 
follow the instructions for \nameref{sec:CreateDatabaseOtherDistributions}.

The selectable database backends depend on the \package{bareos-database-*} packages installed.

For details see \nameref{sec:dbconfig}.

\subsection{Other Platforms}
    \label{sec:CreateDatabaseOtherDistributions}

\subsubsection{PostgreSQL}
If your are using PostgreSQL and your PostgreSQL administration user is \user{postgres} (default), use following commands:

\begin{commands}{Setup Bareos catalog with PostgreSQL}
su postgres -c /usr/lib/bareos/scripts/create_bareos_database
su postgres -c /usr/lib/bareos/scripts/make_bareos_tables
su postgres -c /usr/lib/bareos/scripts/grant_bareos_privileges
\end{commands}


\subsubsection{MySQL}
Make sure, that \user{root} has direct access to the local MySQL server. 
Check if the command \command{mysql} connects to the database without defining the password.
This is the default on RedHat and SUSE distributions. 
On other systems (Debian, Ubuntu),
create the file \file{~/.my.cnf} with your authentication informations:

\begin{config}{MySQL credentials file .my.cnf}
[client]
host=localhost
user=root
password=<input>YourPasswordForAccessingMysqlAsRoot</input>
\end{config}

It is recommended, to secure the Bareos database connection with a password.
See \ilink{Catalog Maintenance -- MySQL}{catalog-maintenance-mysql} about how to archieve this.
For testing, using a password-less MySQL connection is probable okay.
Setup the Bareos database tables by following commands:
\begin{commands}{Setup Bareos catalog with MySQL}
/usr/lib/bareos/scripts/create_bareos_database
/usr/lib/bareos/scripts/make_bareos_tables
/usr/lib/bareos/scripts/grant_bareos_privileges
\end{commands}

As some Bareos updates require a database schema update,
therefore the file \file{/root/.my.cnf} might also be useful in the future.


\section{Start the daemons}
    \label{sec:StartDaemons}

\begin{commands}{Start the Bareos Daemons}
service bareos-dir start
service bareos-sd start
service bareos-fd start
\end{commands}

You will eventually have to allow access to the ports 9101-9103, used by Bareos.

Now you should be able to access the director using the bconsole.

\chapter{Updating Bareos}
\label{bareos-update}

In most cases, a Bareos update is simply done by a package update of the distribution.
Please remind, that Bareos Director and Bareos Storage Daemon must always have the same version.
The version of the File Daemon may differ, see chapter about \ilink{backward compatibility}{backward-compability}.

\section{Updating the database schema}

Sometimes improvements in Bareos make it neccessary to update the database scheme.

\warning{If the Bareos catalog database has not the current schema, the Bareos Director refuses to start.}

Detailed information can than be found in the log file \logfileUnix.

Take a look in the \ilink{Release Notes}{releasenotes} to see, what Bareos updates to require a database schema update.


\subsection{Debian based Linux Distributions}

Since Bareos \sinceVersion{dir}{dbconfig-common (Debian)}{14.2.0} the Debian (and Ubuntu) based packages support the \package{dbconfig-common} mechanism to create and update the Bareos database.
If this is properly configured, the database schema will be automatically adapted by the Bareos packages.

\warning{When using the PostgreSQL backend and updating to Bareos $<$ 14.2.3, it is necessary to manually grant database permissions, normally by}
\begin{commands}{}
<command> </command><parameter>su - postgres -c /usr/lib/bareos/scripts/grant_bareos_privileges</parameter>
\end{commands}
For details see \nameref{sec:dbconfig}.

If you disabled the usage of \package{dbconfig-common}, 
follow the instructions for \nameref{sec:UpdateDatabaseOtherDistributions}.

\subsection{Other Platforms}
    \label{sec:UpdateDatabaseOtherDistributions}

This has to be done as database administrator.
On most platforms Bareos knows only about the credentials to access the Bareos database,
but not about the database administrator to modify the database schema.

The task of updating the database schema is done by the script
\command{/usr/lib/bareos/scripts/update_bareos_tables}.

However, this script requires administration access to the database.
Depending on your distribution and your database, this requires different preparations.
More details can be found in chapter \ilink{Catalog Maitenance}{CatMaintenanceChapter}.

\warning{If you're updating to Bareos $<=$ 13.2.3 and had configured the Bareos database during install using Bareos environment variables (\variable{db_name}, \variable{db_user} or \variable{db_password}, see \nameref{CatMaintenanceChapter}), make sure to have these variables definied in the same way when calling the update and grant scripts. Newer versions of Bareos read this variables from the Director configuration file \configFileDirUnix. However, make sure, the user running the database scripts has read access to this file (or set the environment variables). The \user{postgres} user normally does not have the required permissions.}

\subsubsection{PostgreSQL}
If your are using PostgreSQL and your PostgreSQL administrator is \user{postgres} (default), use following commands:

\begin{commands}{Update PostgreSQL database schema}
su postgres -c /usr/lib/bareos/scripts/update_bareos_tables
su postgres -c /usr/lib/bareos/scripts/grant_bareos_privileges
\end{commands}

The \command{grant_bareos_privileges} command is required, if new databases tables are introduced. It does not hurt to run es multiple times.

After this, restart the Bareos Director and verify it starts without problems.

\subsubsection{MySQL}
Make sure, that \user{root} has direct access to the local MySQL server.
Check if the command \command{mysql} without parameter connects to the database.
If not, you may be required to adapt your local MySQL config file \file{~/.my.cnf}.
It should look similar to this:

\begin{config}{MySQL credentials file .my.cnf}
[client]
host=localhost
user=root
password=<input>YourPasswordForAccessingMysqlAsRoot</input>
\end{config}

If you are able to connect via the \command{mysql} to the database, run the following script from the Unix prompt:
\begin{commands}{Update MySQL database schema}
/usr/lib/bareos/scripts/update_bareos_tables
\end{commands}

Currently on MySQL is it not neccessary to run \command{grant_bareos_privileges}, because access to the database is already given using wildcards.

After this, restart the Bareos Director and verify it starts without problems.
