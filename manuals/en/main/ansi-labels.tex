
\section{Tape Labels: ANSI or IBM}
\label{AnsiLabelsChapter}
\index[general]{Label!Tape Labels}
\index[general]{Tape!Label!ANSI}
\index[general]{Tape!Label!IBM}

By default, Bareos uses its own tape label (see \nameref{backward-compability-tape-format} and \linkResourceDirective{Dir}{Pool}{Label Type}).
However, Bareos also supports reading and write ANSI and IBM tape labels.

\subsection{Reading}

Reading ANSI/IBM labels is important,
if some of your tapes are used by other programs that also support ANSI/IBM labels.
For example, LTFS tapes \index[general]{Tape!LTFS}
are indicated by an ANSI label.

If your are running Bareos in such an environment,
you must set \linkResourceDirective{Sd}{Device}{Check Labels} to yes,
otherwise Bareos will not recognize that these tapes are already in use.

\subsection{Writing}

To configure Bareos to also write ANSI/IBM tape labels,
use \linkResourceDirective{Dir}{Pool}{Label Type}
 or \linkResourceDirective{Sd}{Device}{Label Type}.
With the proper configuration, you can
force Bareos to require ANSI or IBM labels.

Even though Bareos will recognize and write ANSI and IBM labels,
it always writes its own tape labels as well.

If you have labeled your volumes outside of Bareos, then the
ANSI/IBM label will be recognized by Bareos only if you have created
the HDR1 label with {\bf BAREOS.DATA} in the filename field (starting
with character 5).  If Bareos writes the labels, it will use
this information to recognize the tape as a Bareos tape.  This allows
ANSI/IBM labeled tapes to be used at sites with multiple machines
and multiple backup programs.
