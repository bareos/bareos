%%
%%

The Bareos project provides and supports packages that have been released unter
\url{http://download.bareos.org/bareos/release/}

However, the following tabular gives an overview, what components are expected on which plattforms to run:

\begin{tabular}[h]{|l|l|c|c|c|}
  \hline
  \textbf{Operating Systems} & \textbf{Version} & \textbf{Client Daemon} & \textbf{Director Daemon} & \textbf{Storage Daemon} \\
  \hline
  \hline
  GNU/Linux  & all & v12.4 & v12.4 & v12.4 \\
  \hline
  Univention Corporate Linux & App Center & v12.4 & v12.4 & v12.4 \\
  \hline
  \hline
  MS Windows 32bit & 7/8/10       & v12.4 & \elink{nightly}{http://download.bareos.org/bareos/experimental/nightly/windows/} & \elink{nightly}{http://download.bareos.org/bareos/experimental/nightly/windows/} \\
  ~                & 2008/Vista   &  &  &  \\
  ~                & XP           &  &         &  \\
  \hline
  MS Windows 64bit & 7/2012/8/10  & v12.4 & \elink{nightly}{http://download.bareos.org/bareos/experimental/nightly/windows/} & \elink{nightly}{http://download.bareos.org/bareos/experimental/nightly/windows/} \\
  ~                & 2008/Vista   &  &  &  \\
  \hline
  \hline
  MacOS X/Darwin   & ~ & \elink{beta 13.2}{http://download.bareos.org/bareos/beta/13.2/macosx/} &  &  \\
  \hline
  Solaris          & $\geq$ 8 & X & X & X \\
  \hline
  OpenSolaris      & ~ & X & X & X \\
  \hline
  Gentoo           
  \index[general]{Platform!Gentoo}
                    & ~ & \elink{X}{https://packages.gentoo.org/package/app-backup/bareos} & \elink{X}{https://packages.gentoo.org/package/app-backup/bareos} & \elink{X}{https://packages.gentoo.org/package/app-backup/bareos} \\
  \hline
  FreeBSD          & $\geq$ 5.0 & X & X & X  \\
  \hline
  OpenBSD          & ~ & X &  & ~ \\
  \hline
  NetBSD           & ~ & X &  & ~ \\
  \hline
  AIX              & $\geq$ 4.3 & X & $\star$ & $\star$ \\
  \hline
  Irix             & ~ & $\star$ & ~ & ~ \\
  \hline
  True64           & ~ & $\star$ & ~ & ~ \\
  \hline
  BSDI             & ~ & $\star$ & ~ & ~ \\
  \hline
  HPUX             & ~ & $\star$ & ~ & ~ \\
  \hline
\end{tabular}

\begin{tabular}[h]{l l}
\textbf{vVV.V}   & starting with Bareos version VV.V, this platform is official supported by the Bareos.org project \\
\textbf{nightly} & provided by Bareos nightly build. Bug reports are welcome, however it is not official supported \\
\textbf{X}       & known to work \\
\textbf{$\star$} & has been reported to work by the community\\
\end{tabular}

\paragraph{Notes}

\begin{itemize}
    \item by GNU/Linux, we mean all x86 (32/64bit) versions of CentOS, Debian, Fedora, openSUSE, Red Hat Enterprise Linux, SLES and Ubuntu that are officual supported  by the distribution itself.
\end{itemize}

\section{}

\subsection{Packages for the different Linux platforms}
\label{sec:packages}

The following tables show what packages are included for the different platforms with Bareos on \url{http://download.bareos.org/bareos/release/15.2/}.
Bareos tries to provide all packages for all platforms.

On extra packages, it depends if the distribution contains the required dependencies.

Packages names not containing the word \name{bareos} are required packages where we decided to include them ourselfs.

{
    \small
    \input{autogenerated/bareos-15.2-packages-table-redhat.tex}
    \input{autogenerated/bareos-15.2-packages-table-suse.tex}
    \input{autogenerated/bareos-15.2-packages-table-debian.tex}
}



\subsection{Univention Corporate Server}
\index[general]{Platform!Univention Corporte Server}
The Bareos version for the Univention App Center integraties into the Univention Enterprise Linux environment, making it easy to backup all the systems managed by the central Univention Corporate Server, see \url{http://www.bareos.org/en/HOWTO/articles/bareos-univention-documentation.html}.

\subsubsection{Preamble}

The \elink{Univention Corporate Server}{http://www.univention.de/} is an enterprise Linux distribution based on Debian. It consists of an integrated management system for the centralised administration of servers, computer workplaces, users and their rights as well as a wide range of server applications. It also includes an App Center for the easy installation and management of extensions and appliances.

Bareos is part of the Unvention App Center and therefore an Univention environment can easily be extended to provide backup functionality for the Univention servers as well as for the connected client systems. Using the Univention Management Console (UMC), you can also create backup jobs for client computers (Windows or Linux systems), without the need of editing configuration files.

The Bareos app is shipped with a default configuration for the director daemon and the storage daemon. As a result Bareos is able to backup your Univention server without manual configuration.

\textbf{Attention:} You need to review some Univention configuration registry (UCR) variables. Most likely, you will want to set the location where the backups are stored. Otherwise, you may quickly run out of disk space on your backup server!

You will find further information under \nameref{subsubsec:BackupStorage}.

\subsubsection{Quick Start}

 \begin{itemize}
  \item Determine the space requirements and where to store your backup data
  \item Set the UCR variable \parameter{bareos/max_full_volumes} to limit the space used for full backups (integer value, 10 GB-units)
  \item Set the UCR variable \parameter{bareos/max_diff_volumes} to limit the space used for differential backups (integer value, 10 GB-units)
  \item Set the UCR variable \parameter{bareos/max_incr_volumes} to limit the space used for incremental backups (integer value, 10 GB-units)
  \item If an external storage is being used, mount it on \path|/var/lib/bareos/storage|
  \item If you want the backup server to back up itself, set UCR variable \parameter{bareos/backup_myself} to \argument{yes} and reload the director daemon.
  \item Restart \command{bareos-dir}, \command{bareos-fd} and \command{bareos-sd} (or simply reboot the server)
  \item Install the Bareos file daemon on clients and copy configuration file from \file{/etc/bareos/autogenerated/client-configs/<hostname>.conf}
  \item Enable backup jobs for clients in the Univention Management Console
 \end{itemize}


\subsubsection{Setup}

After installation of the Bareos app, Bareos is ready for operation. A default configuration is created automatically.

% Bareos consists of three daemons called \command{director} (or \command{bareos-dir}), \command{storage-daemon} (or \command{bareos-sd}) and \command{filedaemon} (or \command{bareos-fd}). All three daemons are started right after the installation by the univention app center.

% If you want to enable automatic backups of the server, you need to set the Univention configuration registry (UCR) variable \parameter{bareos/backup_myself} to \argument{yes} and reload the director daemon.

$\;$\\
Example:

\begin{commands}{Automatic creation of a default configuration}
root@ucs:~# <input>ucr set bareos/backup_myself='yes'</input>
Setting bareos/backup_myself
File: /etc/bareos/bareos-dir.conf
File: /etc/bareos/bareos-sd.conf
root@ucs:~# <input>/etc/init.d/bareos-dir restart</input>
Stopping Bareos Director: bareos-dir
Starting Bareos Director: bareos-dir.
\end{commands}


\subsubsection{Administration}

The administration takes place in the \bcommand{bconsole}:

\begin{commands}{Starting the bconsole}
root@ucs:~# <input>bconsole</input>
Connecting to Director ucs.antares.de:9101
1000 OK: ucs.antares.de-dir Version: 12.4.3 (15 April 2013)
Enter a period to cancel a command.
*
\end{commands}



\subsubsection{Backup Schedule}

As a result of the default configuration located at the \command{bareos-dir}, the backup schedule will look as follows:

\begin{itemize}
 \item Full Backups are written on the first saturday at 01:00 o'clock
 \item Full Backups are written into the "Full" pool
 \item Full Backups are kept for 365 days
 \item Differential Backups are written on every 2nd to 5th saturday at 01:00 o'clock
 \item Differential Backups are written into the "Differential" pool
 \item Differential Backups are kept for 90 days
 \item Incremental Backups are written every day from monday to friday at 01:00 o'clock
 \item Incremental Backups are written into the "Incremental" pool
 \item Incremental Backups are kept for 30 days
\end{itemize}

That means full backups will be written every first saturday at 01:00 o'clock, differential backups every 2nd to 5th saturday at 01:00 o'clock and incremental backups from monday to friday at 01.00 o'clock. So you have got one full backup every month, four weekly differential and 20 daily differential backups per month.

This schedule is active for the univention server backup of itself and all other clients, which are backed up through the \command{bareos-dir} on the univention server.

There is also a special backup task, which is the Bareos backup of itself for a possible disaster recovery. This backup has got its own backup cycle which starts after the main backups. The backup consists of a database backup for the metadata of the Bareos backup server and a backup of the Bareos configuration files under \path|/etc|.



\subsubsection{Backup data management}

Data from the backup jobs is written to volumes, which are organized in pools (see chapter \nameref{DirectorResourcePool}).

The default configuration uses three different pools, called "Full", "Differential" and "Incremental", which are used for full backups, differential and incremental backups, respectively.

Each pool has a maximum size, which is controlled by the Univention configuration registry (UCR) variables \parameter{bareos/max_full_volumes}, \parameter{bareos/max_diff_volumes} and \parameter{bareos/max_incr_volumes}. Each variable is an integer number specifying the maximum number of volumes in the corresponding pool. Each volume has a maximum size of 10 Gigabytes.

The default maximum number of volumes for each pool is 1, so the maximum disk space used for all backup data is 30 GB.

If you change the UCR variables, the configuration files will be rewritten automatically. After each change you will need to reload the director daemon.

\paragraph{Example for changing the "Full" pool size to 100 GB:}$\;$

\begin{commands}{}
root@ucs:~# <input>ucr set bareos/max_full_volumes='10'</input>
Setting bareos/max_full_volumes
File: /etc/bareos/bareos-dir.conf
File: /etc/bareos/bareos-sd.conf
root@ucs:~# <input>/etc/init.d/bareos-sd restart</input>
Stopping Bareos Storage Daemon: bareos-sd
Starting Bareos Storage Daemon: bareos-sd
\end{commands}


\subsubsection{Backup Storage}
\label{subsubsec:BackupStorage}

\textbf{Attention:} Using the default configuration, Bareos will store backups on your local disk. You may want to store the data to another location to avoid using up all of your disk space.

The location for backups is \path|/var/lib/bareos/storage| in the default configuration.

For example, to use a NAS device for storing backups, you can mount your NAS volume via NFS on \path|/var/lib/bareos/storage|. Alternatively, you can mount the NAS volume to another directory of your own choice, and change the UCR variable \parameter{bareos/filestorage} to the corresponding path.
The directory needs to be writable by user \user{bareos}.

\paragraph{Example for changing the storage path:}$\;$

\begin{commands}{}
root@ucs:/etc/bareos# <input>ucr set bareos/filestorage='/path/to_your/storage'</input>
Setting bareos/filestorage
File: /etc/bareos/bareos-dir.conf
File: /etc/bareos/bareos-sd.conf
\end{commands}

\textbf{Attention:} You need to restart the Bareos storage daemon after having changed the storage path.

\begin{commands}{}
root@ucs:/# <input>/etc/init.d/bareos-sd restart</input>
\end{commands}


\subsubsection{Client and backup job management}
\paragraph{Add a client to the backup setup}$\;$

The univention Bareos application comes with an automatism for the client and job configuration. If you want to add a client to the Bareos director configuration, you need to set the checkbox to true, as you can see in the screenshot below.

\begin{center}
  \includegraphics[width=0.60\textwidth]{\idir bareos-univention}
\end{center}

After having enabled the Bareos backup for a client, it will be configured automatically and loaded into the configuration. Therefore Bareos comes with a special cronjob called \command{univention-bareos}, which performs a restart every day at 00:30 o'clock (Remember: backups will be started at one o'clock at night!).

So if you add a client to the backup at \file{client.conf}, the connection and job data are created, also the corresponding \file{bareos-fd.conf} will be generated to place them on the client you want to backup (you also need to install the bareos-fd client on the client which is to be backed up).

\paragraph{Client and job configuration}$\;$

All clients will be listed in the \file{/etc/bareos/autogenerated/clients.include} which points to a \file{/etc/bareos/autogenerated/clients/xxx.conf}. If you disable the Bareos backup for a client, the client will not be removed from the configuration files. Only the backup job will be set inactive.

\begin{commands}{}
root@ucs:<input>/etc/bareos/autogenerated# vi clients.include </input>
@/etc/bareos/autogenerated/clients/testw4.antares.de.include
@/etc/bareos/autogenerated/clients/testw.antares.de.include
@/etc/bareos/autogenerated/clients/testw2.antares.de.include
\end{commands}

\begin{commands}{}
root@ucs:<input>/etc/bareos/autogenerated/clients# ls -la</input>
insgesamt 28
drwxr-xr-x 2 root root 4096 21. Mai 14:46 .
drwxr-xr-x 5 root root 4096 21. Mai 14:50 ..
-rw-r--r-- 1 root root 430 16. Mai 15:15 generic.template
-rw-r----- 1 root bareos 518 21. Mai 14:49 testw2.antares.de.include
-rw-r----- 1 root bareos 518 16. Mai 18:17 testw4.antares.de.include
-rw-r----- 1 root bareos 513 21. Mai 14:46 testw.antares.de.include
-rw-r--r-- 1 root root 439 16. Mai 15:15 windows.template
\end{commands}

The settings for each job resource are set by the job definition from the bareos-director default configuration and the template files you see above. The client configuration file contains, as you can see below, the connection information and the job information:

\begin{commands}{}
root@ucs:/etc/bareos/autogenerated/clients# vi testw2.antares.de.include
Client {
 Name = "testw2.antares.de-fd"
 Address = "testw2.antares.de"
 Password = "DBLtVnRKq5nRUOrnB3i3qAE38SiDtV8tyhzXIxqR"
 File Retention = 30 days # 30 days
 Job Retention = 6 months # six months
 AutoPrune = no # Prune expired Jobs/Files
}
Job {
 Name = "Backup-testw2.antares.de" #job name
 Client = "testw2.antares.de-fd" # client name
 JobDefs = "DefaultJob" # job definition for the job
 FileSet = "Windows All Drives" # FileSet (data which is backed up)
 Schedule = "WeeklyCycle" # schedule for the backup tasks
 Enabled = "Yes" #this is the ressource which is toggled on/off by enabling or disabling a backup from the univention gui
 }
\end{commands}



\subsection{Debian.org / Ubuntu Universe}
\index[general]{Platform!Debian!Debian.org}
\index[general]{Platform!Debian!8}
\index[general]{Platform!Ubuntu!Universe}
\index[general]{Platform!Ubuntu!Universe!15.04}
\label{sec:DebianOrg}

The distributions of Debian $>=$ 8 include a version of Bareos.
Ubuntu Universe $>=$ 15.04 does also include these packages. 

In the further text, these version will be named \name{Bareos (Debian.org)} 
(also for the Ubuntu Universe version, as this is based on the Debian version).

The source of these packages comes from a seperate branch.
For the release \name{bareos-14.2} this is
\url{https://github.com/bareos/bareos/tree/bareos-14.2-debian} instead of the standard branch \url{https://github.com/bareos/bareos/tree/bareos-14.2}.

The Bareos project tries to limit the differences between these branches to a minimum.

\subsubsection{Limitations of the Debian.org/Ubuntu Universe version of Bareos}
\label{sec:DebianOrgLimitations}

    \begin{itemize}
        \item Debian.org does not include the libfastlz compression library and thesefore the Bareos (Debian.org) packages do not offer the fileset options \parameter{compression=LZFAST}, \parameter{compression=LZ4} and \parameter{compression=LZ4HC}.
        \item Debian.org prefers that Bareos (Debian.org) is linked against GnuTLS instead of OpenSSL. Therefore, the Bareos (Debian.org) package only support \nameref{sec:TransportEncryption} but no \nameref{DataEncryption}.
    \end{itemize}
